\documentclass[letterpaper,12pt]{article}

\usepackage{threeparttable}
\usepackage{geometry}
\geometry{letterpaper,tmargin=1in,bmargin=1in,lmargin=1.25in,rmargin=1.25in}
\usepackage[format=hang,font=normalsize,labelfont=bf]{caption}
\usepackage{amsmath}
\usepackage{multirow}
\usepackage{array}
\usepackage{delarray}
\usepackage{amssymb}
\usepackage{amsthm}
\usepackage{lscape}
\usepackage{natbib}
\usepackage{setspace}
\usepackage{float,color}
\usepackage[pdftex]{graphicx}
\usepackage{mathrsfs}  
\usepackage{pdfsync}
\usepackage{verbatim}
\usepackage{placeins} \usepackage{geometry}
\usepackage{pdflscape}
\synctex=1
\usepackage{hyperref}
\hypersetup{colorlinks,linkcolor=red,urlcolor=blue,citecolor=red}
\usepackage{bm}
\usepackage{amssymb}


\theoremstyle{definition}
\newtheorem{theorem}{Theorem}
\newtheorem{acknowledgement}[theorem]{Acknowledgement}
\newtheorem{algorithm}[theorem]{Algorithm}
\newtheorem{axiom}[theorem]{Axiom}
\newtheorem{case}[theorem]{Case}
\newtheorem{claim}[theorem]{Claim}
\newtheorem{conclusion}[theorem]{Conclusion}
\newtheorem{condition}[theorem]{Condition}
\newtheorem{conjecture}[theorem]{Conjecture}
\newtheorem{corollary}[theorem]{Corollary}
\newtheorem{criterion}[theorem]{Criterion}
\newtheorem{definition}{Definition} % Number definitions on their own
\newtheorem{derivation}{Derivation} % Number derivations on their own
\newtheorem{example}[theorem]{Example}
\newtheorem*{exercise}{Exercise} % Number exercises on their own
\newtheorem{lemma}[theorem]{Lemma}
\newtheorem{notation}[theorem]{Notation}
\newtheorem{problem}[theorem]{Problem}
\newtheorem{proposition}{Proposition} % Number propositions on their own
\newtheorem{remark}[theorem]{Remark}
\newtheorem{solution}[theorem]{Solution}
\newtheorem{summary}[theorem]{Summary}
\bibliographystyle{aer}
\newcommand\ve{\varepsilon}
\renewcommand\theenumi{\roman{enumi}}

\title{Math Sec 3.1}
\author{Rex McArthur\\Math 344}


\begin{document}
\maketitle
\exercise{3.31}\\
$\Leftarrow$ Suppose that $a^{p} = b^{q}$. Then $(a^{p}/p+b^{q}/q)=a^{p}(1/p+1/q)=a^{p}$ and $ab = (a^{p})^{1/p}(b^{q})^{1/q}=(a^{p})^{1/p}(a^{p})^{1/q}=a^{p}$, showing that $ab = a^{p}/p+b^{q}/q$. \\ \\
$\Rightarrow$ Suppose that $ab=(1/p)a^{p}+(1/q)b^{q}$ Upon dividing by $ab$ and using the fact that $a^{p}/b=(a^{p}/b^{q})^{1/q}$ and $b^{q}/a=(b^{q}/a^{p})^{1/p}$, we see that \[ \frac{1}{p}(\frac{a^{p}}{b^{q}})^{1/q}+\frac{1}{q}(\frac{b^{q}}{a^{p}})^{1/p}=1\]
Let $x= a^{p}/b^{q}$. Multiplying by $x^{1/p}$, we obtain 
\[ \frac{1}{p}x^{1/p+1/q}+\frac{1}{q}=1\]
 \[\rightarrow \frac{1}{p}x+\frac{1}{q}=1\]
 This implies that $x= a^{p}/b^{q}=1$, so $a^{p}= b^{q}$, as desired.\\\\

\exercise{3.32}\\
Suppose $\epsilon \geq 1$. Thus, by Young's inequality,
\begin{align*}
    ab & \leq \frac{a^2}{2} + \frac{b^2}{2} \\
    & \leq \frac{\epsilon ^2}{\epsilon}\big( \frac{a^2}{2} + \frac{b^2}{2} \big)\\
    & \leq  \frac{a^2 + \epsilon^2 b^2}{2\epsilon} \\
    ab & \leq \frac{a^2}{2\epsilon} + \frac{\epsilon b^2}{2} \\
\end{align*}
Now, suppose $\epsilon < 1$. Thus, by Young's inequality, 
\begin{align*}
    ab & \leq \frac{a^2}{2} + \frac{b^2}{2} \\
    & \leq \frac{1}{\epsilon}\big( \frac{a^2 + b^2}{2} \big)\\
    & \leq \frac{a^2}{2\epsilon} + \frac{b^2}{2\epsilon} \\
    ab & \leq \frac{a^2}{2\epsilon} + \frac{\epsilon b^2}{2} \\
\end{align*}
In all cases, $ab \leq \frac{a^2}{2\epsilon} + \frac{\epsilon b^2}{2}$\\

\exercise{3.33}\\
We note first that if $a=b$, 
\begin{align*}
    a^\theta b^{1-\theta} =a^\theta a^{1-\theta} = a = \theta a + (1-\theta)a = \theta a + (1-\theta)b
\end{align*}
Now suppose $a \neq b$
\begin{align*}
a^\theta b^{1-\theta} &\leq \theta a + (1-\theta)b\\
\ln(a^\theta b^{1-\theta}) &\leq \ln(\theta a + (1-\theta)b)
\end{align*}
Note that, because the natural log is convex, $\theta \ln(a) < \ln(\theta a)$. Thus,
\begin{align*}
    \ln(a^\theta b^{1-\theta}) &= \theta \ln(a) + (1-\theta)\ln(b)\\
    & < \ln(\theta a ) + \ln( (1-\theta) b) \\
    & < \ln( \theta a + (1- \theta)b) 
\end{align*}
Thus, equality holds if and only if $a = b$.

\exercise{3.34}\\
Letting $\theta=\frac{1}{2}$, we get : 
\begin{align*}
&a^\frac{1}{2}b^\frac{1}{2} \leq \frac{1}{2} (a+b) \\
&(ab)^\frac{1}{2} \leq \frac{1}{2} (a+b) \\
&(\text{area})^\frac{1}{2} \leq \frac{1}{4} \text{Perimeter} \\
& P \geq 4 \sqrt{A}
\end{align*}
The minimum lies at $P=4 \sqrt{A}$ which holds only for 
\[2(a+b)=4\sqrt{ab} \Rightarrow 4(a+b)^2 = 16(ab) \Rightarrow 4(a-b)^2=0 \Rightarrow a=b \]

\exercise{3.35}\\
For arbitrary dimensions, we may extend the Arithmetic Geometric Mean inequality to say,
\begin{align*}
    (x_1 \cdot ... \cdot x_n)^{\frac{1}{n}} \leq \frac{x_1+\dots+x_n}{n}
\end{align*}
Where equality holds if and only if each $x_i$ is equal to all the others.\\
The n-dim. cube must have n verticies, and $2^{n-1}$ edges, because each vertex is connected to n edges, so the total length of all vertecies is going to be $2^{n-1}(x_1+x_2+\dots+x_n)$, where each x is a length of a vertex.
The volume is simply going to be $2^n(x_1\cdot \dots \cdot x_n)^{\frac{1}{n}}$, Thus, the inequality gives us
\begin{align*}
    2^{n}(x_1\cdot ... \cdot x_n)^{\frac{1}{n}} & \leq 2^{n-1} \sum_{i=1}^n x_i\\
    (x_1\cdot ... \cdot x_n)^{\frac{1}{n}} & \leq \frac{1}{n}\sum_{i=1}^n x_i\\
    \text{Area}^{\frac{1}{n}} & \leq \frac{1}{n}\sum_{i=1}^n x_i \\
\end{align*}
The minimum value is found here when these are equal.\\
Note, if $x_i = y$ for all i, then we know
\begin{align*}
    y^n & = \frac{1}{n}\sum_{i=1}^n x_i = \frac{1}{n} (n(y^n)) = y^n
\end{align*}
Otherwise, we know that these won't be equal, by our definition of the Arithmetic Geometric Mean inequality for n dimensions.\\
Thus, the n dimensional rectangle with a fixed area with the least perimeter is going to be the square.

\exercise{3.36}\\







\end{document}

