\documentclass[letterpaper,12pt]{article}

\usepackage{threeparttable}
\usepackage{geometry}
\geometry{letterpaper,tmargin=1in,bmargin=1in,lmargin=1.25in,rmargin=1.25in}
\usepackage[format=hang,font=normalsize,labelfont=bf]{caption}
\usepackage{amsmath}
\usepackage{multirow}
\usepackage{array}
\usepackage{delarray}
\usepackage{amssymb}
\usepackage{amsthm}
\usepackage{lscape}
\usepackage{natbib}
\usepackage{setspace}
\usepackage{float,color}
\usepackage[pdftex]{graphicx}
\usepackage{mathrsfs}  
\usepackage{pdfsync}
\usepackage{verbatim}
\usepackage{placeins} \usepackage{geometry}
\usepackage{pdflscape}
\synctex=1
\usepackage{hyperref}
\hypersetup{colorlinks,linkcolor=red,urlcolor=blue,citecolor=red}
\usepackage{bm}
\usepackage{amssymb}


\theoremstyle{definition}
\newtheorem{theorem}{Theorem}
\newtheorem{acknowledgement}[theorem]{Acknowledgement}
\newtheorem{algorithm}[theorem]{Algorithm}
\newtheorem{axiom}[theorem]{Axiom}
\newtheorem{case}[theorem]{Case}
\newtheorem{claim}[theorem]{Claim}
\newtheorem{conclusion}[theorem]{Conclusion}
\newtheorem{condition}[theorem]{Condition}
\newtheorem{conjecture}[theorem]{Conjecture}
\newtheorem{corollary}[theorem]{Corollary}
\newtheorem{criterion}[theorem]{Criterion}
\newtheorem{definition}{Definition} % Number definitions on their own
\newtheorem{derivation}{Derivation} % Number derivations on their own
\newtheorem{example}[theorem]{Example}
\newtheorem*{exercise}{Exercise} % Number exercises on their own
\newtheorem{lemma}[theorem]{Lemma}
\newtheorem{notation}[theorem]{Notation}
\newtheorem{problem}[theorem]{Problem}
\newtheorem{proposition}{Proposition} % Number propositions on their own
\newtheorem{remark}[theorem]{Remark}
\newtheorem{solution}[theorem]{Solution}
\newtheorem{summary}[theorem]{Summary}
\bibliographystyle{aer}
\newcommand\ve{\varepsilon}
\renewcommand\theenumi{\roman{enumi}}

\title{Math Sec 3.3}
\author{Rex McArthur\\Math 344}


\begin{document}
\maketitle
\exercise{3.12}\\
The Gram Schmidt would yield zero vectors, because they are linearly dependent, and are just linear combinations of one another.

\exercise{3.13}\\
Let $x_1 = \begin{bmatrix} 1 \\ 1 \end{bmatrix}$, and $x_2 = \begin{bmatrix} 1 \\ 0 \end{bmatrix}$\\
Thus, the normalized $x_1$ is $v_1 = \begin{bmatrix} \frac{1}{\sqrt{2}} \\ \frac{1}{\sqrt{2}} 
\end{bmatrix}$ Now applying the Gram-Schmidt, we find that \\
\begin{align}
    \mathbf{p}_1 = \text{proj}_{v_1}(x_2) = \langle \mathbf{v_1}, \mathbf{x_2} \rangle \mathbf{v_1} =  \begin{bmatrix} \frac{1}{2} \\ \frac{1}{2} \end{bmatrix}
\end{align}
Thus, 
\[
q_2 = \frac{\mathbf{x_2} - \mathbf{p_1}}{\|\mathbf{x_2} - \mathbf{p_1}\|} = \begin{bmatrix} \frac{1}{\sqrt{2}} \\ \frac{-1}{\sqrt{2}} \end{bmatrix}
\]
The set of orothonomral vectors are $\big\{ \begin{bmatrix} \frac{1}{\sqrt{2}} \\ \frac{1}{\sqrt{2}}\end{bmatrix}, \begin{bmatrix} \frac{1}{\sqrt{2}} \\ \frac{-1}{\sqrt{2}} \end{bmatrix} \big\}$

\exercise{3.14}\\
consider the set $\{1,x,x^2,x^3\}$. Using the Chebyshev inner product, we have that \\
$u_1 = \frac{1}{\sqrt{\pi}}$. Note, $1, x$ are orthoganal to eachoter, because their inner product is zero, thus we can just normalize $x$ to find\\
$u_2 = \frac{x}{\sqrt{\pi/2}}$
By applying the Gram-Shmidt process on $u_2$ and $x^2$ we find that \\
$v_3 = x^2 - \frac{1}{2}$, and we can normalize that to get
$u_3 = \frac{x^2-\frac{1}{2}}{\sqrt{\frac{\pi}{8}}}$
By applying the Gram-Shmidt process on $u_3$ and $x^3$ we find that,
$v_4 = x^3 - \frac{3}{4}x$, and by normalizing we obtain\\
$u_4 = \frac{x^3 - \frac{3}{4}x}{\sqrt{\frac{7\pi}{8}}}$.\\
Thus, the set of orthonormal basis vectors for this set is \\
$( \frac{1}{\sqrt{\pi}},
\frac{x}{\sqrt{\pi/2}},
\frac{x^2-\frac{1}{2}}{\sqrt{\pi/8}},
\frac{x^3 - \frac{3}{4}x}{\sqrt{\frac{7\pi}{8}}} )$

\end{document}

