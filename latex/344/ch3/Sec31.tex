\documentclass[letterpaper,12pt]{article}

\usepackage{threeparttable}
\usepackage{geometry}
\geometry{letterpaper,tmargin=1in,bmargin=1in,lmargin=1.25in,rmargin=1.25in}
\usepackage[format=hang,font=normalsize,labelfont=bf]{caption}
\usepackage{amsmath}
\usepackage{multirow}
\usepackage{array}
\usepackage{delarray}
\usepackage{amssymb}
\usepackage{amsthm}
\usepackage{lscape}
\usepackage{natbib}
\usepackage{setspace}
\usepackage{float,color}
\usepackage[pdftex]{graphicx}
\usepackage{mathrsfs}  
\usepackage{pdfsync}
\usepackage{verbatim}
\usepackage{placeins} \usepackage{geometry}
\usepackage{pdflscape}
\synctex=1
\usepackage{hyperref}
\hypersetup{colorlinks,linkcolor=red,urlcolor=blue,citecolor=red}
\usepackage{bm}
\usepackage{amssymb}


\theoremstyle{definition}
\newtheorem{theorem}{Theorem}
\newtheorem{acknowledgement}[theorem]{Acknowledgement}
\newtheorem{algorithm}[theorem]{Algorithm}
\newtheorem{axiom}[theorem]{Axiom}
\newtheorem{case}[theorem]{Case}
\newtheorem{claim}[theorem]{Claim}
\newtheorem{conclusion}[theorem]{Conclusion}
\newtheorem{condition}[theorem]{Condition}
\newtheorem{conjecture}[theorem]{Conjecture}
\newtheorem{corollary}[theorem]{Corollary}
\newtheorem{criterion}[theorem]{Criterion}
\newtheorem{definition}{Definition} % Number definitions on their own
\newtheorem{derivation}{Derivation} % Number derivations on their own
\newtheorem{example}[theorem]{Example}
\newtheorem*{exercise}{Exercise} % Number exercises on their own
\newtheorem{lemma}[theorem]{Lemma}
\newtheorem{notation}[theorem]{Notation}
\newtheorem{problem}[theorem]{Problem}
\newtheorem{proposition}{Proposition} % Number propositions on their own
\newtheorem{remark}[theorem]{Remark}
\newtheorem{solution}[theorem]{Solution}
\newtheorem{summary}[theorem]{Summary}
\bibliographystyle{aer}
\newcommand\ve{\varepsilon}
\renewcommand\theenumi{\roman{enumi}}

\title{Math Sec 3.1}
\author{Rex McArthur\\Math 344}


\begin{document}
\maketitle
\exercise{3.1}\\
(i)
\begin{align*}
    \langle \mathbf{x},\mathbf{y} \rangle & = \frac{1}{2} (\langle \mathbf{x},\mathbf{y} \rangle + \langle \mathbf{x},\mathbf{y} \rangle ) \\
    & = \frac{1}{2}(\frac{1}{2}(\langle \mathbf{x},\mathbf{x} \rangle + 2 \langle \mathbf{x},\mathbf{y} \rangle + \langle \mathbf{y},\mathbf{y} \rangle -( \langle \mathbf{x},\mathbf{x} \rangle  - 2 \langle \mathbf{x},\mathbf{y} \rangle -\langle \mathbf{y},\mathbf{y} \rangle ))) \\
    & = \frac{1}{4}( \langle \mathbf{x+y},\mathbf{x+y} \rangle - \langle \mathbf{x-y},\mathbf{x-y} \rangle ) \\
    & = \frac{1}{4} (||x + y||)^2 - (||x-y||^2)
\end{align*}
(ii)
\begin{align*}
    \|x\|^2 + \|y\|^2  &= \frac{1}{2}(\langle \mathbf{x},\mathbf{x} \rangle + \langle \mathbf{x},\mathbf{y} \rangle )     \\
    & = \frac{1}{2}(\langle \mathbf{x},\mathbf{x} \rangle + \langle \mathbf{y},\mathbf{y} \rangle \langle \mathbf{x},\mathbf{x} \rangle + \langle \mathbf{y},\mathbf{y} \rangle )\\
    & = \frac{1}{2}(\frac{1}{2}(\langle \mathbf{x},\mathbf{x} \rangle + 2 \langle \mathbf{x},\mathbf{y} \rangle + \langle \mathbf{y},\mathbf{y} \rangle - (\langle \mathbf{x},\mathbf{y} \rangle - \langle \mathbf{y},\mathbf{y} \rangle )))\\
    & = \frac{1}{4}(\langle \mathbf{x+y},\mathbf{x+y} \rangle - \langle \mathbf{x-y},\mathbf{x-y} \rangle ) \\
    & = \frac{1}{4}(\| x+y\|^2 - \|x-y\|^2)
\end{align*}

\exercise{3.2}\\
\begin{align*}
    \langle \mathbf{x},\mathbf{y} \rangle & = \frac{1}{2}(\langle \mathbf{x},\mathbf{y} \rangle + \langle \mathbf{y},\mathbf{x} \rangle - \langle \mathbf{y},\mathbf{x} \rangle + \langle \mathbf{x},\mathbf{y} \rangle ) \\
    & = 
\end{align*}

\end{document}

