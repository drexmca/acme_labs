\documentclass[letterpaper,12pt]{article}

\usepackage{threeparttable}
\usepackage{geometry}
\geometry{letterpaper,tmargin=1in,bmargin=1in,lmargin=1.25in,rmargin=1.25in}
\usepackage[format=hang,font=normalsize,labelfont=bf]{caption}
\usepackage{amsmath}
\usepackage{multirow}
\usepackage{array}
\usepackage{delarray}
\usepackage{amssymb}
\usepackage{amsthm}
\usepackage{lscape}
\usepackage{natbib}
\usepackage{setspace}
\usepackage{float,color}
\usepackage[pdftex]{graphicx}
\usepackage{mathrsfs}  
\usepackage{pdfsync}
\usepackage{verbatim}
\usepackage{placeins} \usepackage{geometry}
\usepackage{pdflscape}
\synctex=1
\usepackage{hyperref}
\hypersetup{colorlinks,linkcolor=red,urlcolor=blue,citecolor=red}
\usepackage{bm}
\usepackage{amssymb}


\theoremstyle{definition}
\newtheorem{theorem}{Theorem}
\newtheorem{acknowledgement}[theorem]{Acknowledgement}
\newtheorem{algorithm}[theorem]{Algorithm}
\newtheorem{axiom}[theorem]{Axiom}
\newtheorem{case}[theorem]{Case}
\newtheorem{claim}[theorem]{Claim}
\newtheorem{conclusion}[theorem]{Conclusion}
\newtheorem{condition}[theorem]{Condition}
\newtheorem{conjecture}[theorem]{Conjecture}
\newtheorem{corollary}[theorem]{Corollary}
\newtheorem{criterion}[theorem]{Criterion}
\newtheorem{definition}{Definition} % Number definitions on their own
\newtheorem{derivation}{Derivation} % Number derivations on their own
\newtheorem{example}[theorem]{Example}
\newtheorem*{exercise}{Exercise} % Number exercises on their own
\newtheorem{lemma}[theorem]{Lemma}
\newtheorem{notation}[theorem]{Notation}
\newtheorem{problem}[theorem]{Problem}
\newtheorem{proposition}{Proposition} % Number propositions on their own
\newtheorem{remark}[theorem]{Remark}
\newtheorem{solution}[theorem]{Solution}
\newtheorem{summary}[theorem]{Summary}
\bibliographystyle{aer}
\newcommand\ve{\varepsilon}
\renewcommand\theenumi{\roman{enumi}}

\title{Math Sec 3.1}
\author{Rex McArthur\\Math 344}


\begin{document}
\maketitle
\exercise{3.1}\\
(i)
\begin{align*}
    \langle \mathbf{x},\mathbf{y} \rangle & = \frac{1}{2} (\langle \mathbf{x},\mathbf{y} \rangle + \langle \mathbf{x},\mathbf{y} \rangle ) \\
    & = \frac{1}{2}(\frac{1}{2}(\langle \mathbf{x},\mathbf{x} \rangle + 2 \langle \mathbf{x},\mathbf{y} \rangle + \langle \mathbf{y},\mathbf{y} \rangle -( \langle \mathbf{x},\mathbf{x} \rangle  - 2 \langle \mathbf{x},\mathbf{y} \rangle -\langle \mathbf{y},\mathbf{y} \rangle ))) \\
    & = \frac{1}{4}( \langle \mathbf{x+y},\mathbf{x+y} \rangle - \langle \mathbf{x-y},\mathbf{x-y} \rangle ) \\
    & = \frac{1}{4} (||x + y||)^2 - (||x-y||^2)
\end{align*}
(ii)
\begin{align*}
    \|x\|^2 + \|y\|^2  &= \frac{1}{2}(\langle \mathbf{x},\mathbf{x} \rangle + \langle \mathbf{x},\mathbf{y} \rangle )     \\
    & = \frac{1}{2}(\langle \mathbf{x},\mathbf{x} \rangle + \langle \mathbf{y},\mathbf{y} \rangle \langle \mathbf{x},\mathbf{x} \rangle + \langle \mathbf{y},\mathbf{y} \rangle )\\
    & = \frac{1}{2}(\frac{1}{2}(\langle \mathbf{x},\mathbf{x} \rangle + 2 \langle \mathbf{x},\mathbf{y} \rangle + \langle \mathbf{y},\mathbf{y} \rangle - (\langle \mathbf{x},\mathbf{y} \rangle - \langle \mathbf{y},\mathbf{y} \rangle )))\\
    & = \frac{1}{4}(\langle \mathbf{x+y},\mathbf{x+y} \rangle - \langle \mathbf{x-y},\mathbf{x-y} \rangle ) \\
    & = \frac{1}{4}(\| x+y\|^2 - \|x-y\|^2)
\end{align*}

\exercise{3.2}\\
\begin{align*}
    \langle \mathbf{x},\mathbf{y} \rangle & = \frac{1}{2}(\langle \mathbf{x},\mathbf{y} \rangle + \langle \mathbf{y},\mathbf{x} \rangle - \langle \mathbf{y},\mathbf{x} \rangle + \langle \mathbf{x},\mathbf{y} \rangle ) \\
    & = \frac{1}{2}(\langle \mathbf{x}, \mathbf{y} \rangle  + \langle \mathbf{x}, \mathbf{y} \rangle -i^2 \langle \mathbf{x}, \mathbf{y} \rangle  + i^2 \langle \mathbf{y}, \mathbf{x} \rangle ) \\
    & = \frac{1}{2}(\frac{1}{2}(\| x+y \| ^2 - \| x-y\|^2) + \frac{1}{2}(i \langle \mathbf{x}, \mathbf{iy} \rangle - i \langle \mathbf{x}, \mathbf{iy} \rangle \\
    & = \frac{1}{4} (\|x+y\|^2 - \|x-y \| + i\|x-iy\| - i \|x+iy\|^2)
\end{align*}

\exercise{3.3}\\

\begin{align*}
cos(\theta)& = \frac{\langle x,x^{5}\rangle }{\|x\|\|x^{5}\|} \\
\langle x,x^{5}\rangle  &= \int^{1}_{0} x^{6} dx = \frac{1}{7}x^{7} |_{0}^{1} = \frac{1}{7} \\
\|x\| &= \sqrt{\langle x,x\rangle } = \int^{1}_{0} x^{2} dx = \frac{1}{3}x^{3} |_{0}^{1} = \frac{1}{\sqrt{3}} \\ 
\|x^{5}\| &= \sqrt{\langle x,x\rangle } = \int^{1}_{0} x^{10} dx = \frac{1}{11}x^{11} |_{0}^{1} = \frac{1}{\sqrt{11}} \\ 
\theta &= cos^{-1}(\frac{\sqrt{33}}{7}) \approx .60824
\end{align*}
ii)
\begin{align*}
cos(\theta)& = \frac{\langle x^{2},x^{4}\rangle }{\|x^{2}\|\|x^{4}\|} \\
\langle x,x^{4}\rangle  &= \int^{1}_{0} x^{5} dx = \frac{1}{7}x^{7} |_{0}^{1} = \frac{1}{7} \\
\|x^{2}\| &= \sqrt{\langle x,x\rangle } = \int^{1}_{0} x^{4} dx = \frac{1}{5}x^{5} |_{0}^{1} = \frac{1}{\sqrt{5}} \\ 
\|x^{4}\| &= \sqrt{\langle x,x\rangle } = \int^{1}_{0} x^{8} dx = \frac{1}{9}x^{9} |_{0}^{1} = \frac{1}{\sqrt{9}} \\ 
\theta &= cos^{-1}(\frac{\sqrt{45}}{7}) \approx .289
\end{align*}


\exercise{3.4}\\
Suppose $\|T \mathbf{x}\| = a\|\mathbf{x}\|$, Thus,
\begin{align*}
    \frac{\langle \mathbf{Tx}, \mathbf{Ty} \rangle  }{\|x\|\|y\|} & =
    \frac{\frac{1}{4} (||Tx + Ty||)^2 - (||Tx-Ty||^2)}{\|Tx\|\|Ty\|} \\
    & = \frac{\frac{1}{4} a^2 \|x+y\| ^2 - \|x-y\|^2}{a^2 \|x\|\|y\|} \\
    & = \frac{\langle \mathbf{x}, \mathbf{y} \rangle }{ \|x\| \|y\|}
\end{align*}
Now suppose, $\frac{\langle \mathbf{Tx}, \mathbf{Ty} \rangle }{\|Tx\|\|Ty\|} = \frac{\langle \mathbf{x}, \mathbf{y} \rangle }{\|x\|\|y\|} $. Thus, $\langle \mathbf{Tx}, \mathbf{Ty} \rangle = \langle \mathbf{x}, \mathbf{y} \rangle  = 0 $ iff $\mathbf{x},\mathbf{y}$ are orthogonal, thus it preserves angle.
Note that $\mathbf{y} = \text{proj}_x \mathbf{y} + \mathbf{r}$, 
\begin{align*}
    \frac{\langle \mathbf{Tx}, \mathbf{Ty} \rangle 
}{\|T \mathbf{x} \| \| T \mathbf{y}\|} 
& = \frac{\langle \mathbf{Tx}, \mathbf{T(\alpha \mathbf{x} + \mathbf{r})} \rangle }{\|Tx\|\|Ty\|} \\
\end{align*}
Thus, 
\begin{align*}
\frac{\langle \mathbf{Tx}, \mathbf{T(\alpha \mathbf{x} + \mathbf{r})} \rangle }{\|Tx\|\|Ty\|} & = \frac{\langle \mathbf{x}, \mathbf{\alpha \mathbf{x} + \mathbf{r}} \rangle }{\|x\|\|y\|}\\
\frac{\alpha \|Tx\|^2}{\|Tx\|\|Ty\|} & = \frac{\alpha \|x\|^2}{\|x\|\|y\|} \\
\frac{\alpha \|Tx\|}{\|Ty\|} & = \frac{\alpha \|x\|}{\|y\|} \\
\frac{\alpha \|Tx\|}{\|x\|} & = \frac{\alpha \|Ty\|}{\|y\|} \\
\end{align*}
Which is equal to some constant for both, and thus, 
\begin{equation*}
    (\|T \mathbf{x} \| = a \|\mathbf{x}\|)
\end{equation*}
and,
\begin{equation*}
    (\|T \mathbf{y} \| = a \|\mathbf{y}\|)
\end{equation*}

\exercise{3.5}\\
\[f = e^{x} ~~~~~~~~~ g = x-1 \] \\
\begin{align*}
    \text{proj}_{u}(f) &= \frac{\langle g,f\rangle }{\langle g,g\rangle } \cdot g \\
\langle g,f\rangle  &= \langle e^{x}, x-1\rangle  = \int_{0}^{1}xe^{x} - e^{x} dx = xe^{x} - 2e^{x} |_{0}^{1} = (e-2e) + (2) = -e +2 \\ 
\langle g,g\rangle  &= \int_{0}^{1}(x-1)^{2} = \int_{0}^{1} x^{2} - 2x + 1 dx = \frac{1}{3}x^{3} - x^{2} + x |_{0}^{1} = \frac{1}{3} \\
\end{align*}
Thus, we have that projection $e^x$ onto $(x-1)$ is
\begin{align*}
\frac{-e+2}{\frac{1}{3}} &\cdot(x-1) = (-3e+6)(x-1)
\end{align*}

\exercise{3.6}\\
Note,  $0\leq\|x-\lambda y\|^{2}$ and let $\lambda = \frac{\langle x,y\rangle }{\langle y,y\rangle }$.
\[\|x-\lambda y\|^{2} = \langle x-\lambda y, x-\lambda y\rangle \]

Note that $x-\lambda y$ is orthogonal to $y$ and $\lambda y$, because
\begin{equation*}
    \langle \mathbf{x} - \frac{\langle \mathbf{x}, \mathbf{y} \rangle}{\|y\|^2} ,\frac{ -\langle \mathbf{x}, \mathbf{y} \rangle y}{\|y\|^2}  \rangle = \langle \mathbf{x} -\text{proj}_y \mathbf{x} , -\text{proj}_y \mathbf{x} \rangle
\end{equation*}
Which leads to y being orthogonal to both, because it is a projection onto it.
Thus, $\langle x-\lambda y,-\lambda y\rangle  = 0$ and $\langle -\lambda y, x-\lambda y\rangle  = 0$. 
and by Bessel's inequality we get
 \begin{align*}
0 \leq \|x-\lambda y\|^{2} &= \langle x-\lambda y, x-\lambda y\rangle  \\
&= \langle x,x\rangle  - \langle \lambda y, x\rangle  \\
&= \| x^{2} \| - \frac{\langle x,y\rangle ^{2}}{\langle y,y\rangle ^{2}} \\
&= \| x^{2} \| - \frac{\langle x,y\rangle ^{2}}{\|y\|^{2}} \\
0 \leq \| x^{2} \| &- \frac{\langle x,y\rangle ^{2}}{\|y\|^{2}} \\
&\Rightarrow \frac{\langle x,y\rangle ^{2}}{\|y\|^{2}} \leq \| x^{2} \| \\
&\Rightarrow \frac{\langle x,y\rangle }{\|y\|} \leq \| x \| \\
&\Rightarrow \langle x,y\rangle  \leq \| x \| ~\| y \| \\
\end{align*}

\end{document}

