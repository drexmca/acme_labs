\documentclass[letterpaper,12pt]{article}

\usepackage{threeparttable}
\usepackage{geometry}
\geometry{letterpaper,tmargin=1in,bmargin=1in,lmargin=1.25in,rmargin=1.25in}
\usepackage[format=hang,font=normalsize,labelfont=bf]{caption}
\usepackage{amsmath}
\usepackage{multirow}
\usepackage{array}
\usepackage{delarray}
\usepackage{amssymb}
\usepackage{amsthm}
\usepackage{lscape}
\usepackage{natbib}
\usepackage{setspace}
\usepackage{float,color}
\usepackage[pdftex]{graphicx}
\usepackage{mathrsfs}  
\usepackage{pdfsync}
\usepackage{verbatim}
\usepackage{placeins} \usepackage{geometry}
\usepackage{pdflscape}
\synctex=1
\usepackage{hyperref}
\hypersetup{colorlinks,linkcolor=red,urlcolor=blue,citecolor=red}
\usepackage{bm}
\usepackage{amssymb}


\theoremstyle{definition}
\newtheorem{theorem}{Theorem}
\newtheorem{acknowledgement}[theorem]{Acknowledgement}
\newtheorem{algorithm}[theorem]{Algorithm}
\newtheorem{axiom}[theorem]{Axiom}
\newtheorem{case}[theorem]{Case}
\newtheorem{claim}[theorem]{Claim}
\newtheorem{conclusion}[theorem]{Conclusion}
\newtheorem{condition}[theorem]{Condition}
\newtheorem{conjecture}[theorem]{Conjecture}
\newtheorem{corollary}[theorem]{Corollary}
\newtheorem{criterion}[theorem]{Criterion}
\newtheorem{definition}{Definition} % Number definitions on their own
\newtheorem{derivation}{Derivation} % Number derivations on their own
\newtheorem{example}[theorem]{Example}
\newtheorem*{exercise}{Exercise} % Number exercises on their own
\newtheorem{lemma}[theorem]{Lemma}
\newtheorem{notation}[theorem]{Notation}
\newtheorem{problem}[theorem]{Problem}
\newtheorem{proposition}{Proposition} % Number propositions on their own
\newtheorem{remark}[theorem]{Remark}
\newtheorem{solution}[theorem]{Solution}
\newtheorem{summary}[theorem]{Summary}
\bibliographystyle{aer}
\newcommand\ve{\varepsilon}
\renewcommand\theenumi{\roman{enumi}}

\title{Math Sec 5.1}
\author{Rex McArthur\\Math 344}


\begin{document}
\maketitle
\exercise{5.1}\\
It is sufficent to show in each case, Positive definiteness, symmetry, and the triangle inequality.
\begin{enumerate}
    \item Positive Definiteness: $d_1(\textbf{x}, \textbf{y}) \geq 0$ and $d_2(\textbf{x}, \textbf{y}) \geq 0$ thus, the sum of two numbers greater than 0 is greater than 0.
\\
Symmetry: $(d_1+d_2)(\textbf{x}, \textbf{y}) = d_1(\textbf{x}, \textbf{y}) + d_2(\textbf{x}, \textbf{y}) = d_1(\textbf{y}, \textbf{x}) + d_2(\textbf{y}, \textbf{x}) = (d_1+d_2)(\textbf{y}, \textbf{x})$\\
Triangle Inequality: We know that the triangle inequality holds for $d_1$ and $d_2$ respectivly. Thus we can say,
\begin{align*}
    (d_1 + d_2)(\textbf{x}, \textbf{y}) &= d_1(\textbf{x}, \textbf{y}) + d_2(\textbf{x}, \textbf{y}) \\
    & \leq d_1(\textbf{x}, \textbf{z}) + d_1(\textbf{z}, \textbf{y}) + d_2(\textbf{x}, \textbf{z}) + d_2(\textbf{z}, \textbf{y}) \\
    &= (d_1(\textbf{x}, \textbf{z}) + d_2(\textbf{x}, \textbf{z})) + (d_1(\textbf{z}, \textbf{y}) + d_2(\textbf{z}, \textbf{y})) \\
    &= (d_1 + d_2)(\textbf{x}, \textbf{z}) + (d_1 + d_2)(\textbf{z}, \textbf{y}) 
\end{align*}
    \item This is not a metric. Conterexample, let $d_2 = 5$ and $d_1 = 3$ be our metrics, $d_1 - d_2 = -2$ which violates Postivite definiteness, therfore, it is not a metric.
    \item WLOG we can say our metric is $d_1$ which exhibits all the properties of a metric, therefore, $min(d_1, d_2)$ is a metric.
    \item WLOG we can say our metric is $d_1$ which exhibits all the properties of a metric, therefore, $max(d_1, d_2)$ is a metric.
\end{enumerate}

\exercise{5.2}\\
The metric could be named thus, because to get to any city, we would have to be routed through Paris, unless the city lies on a direct route between the origin and Paris.\\
To show it is a metric, consider two cases, \\
Case 1: $\textbf{x}=\alpha \textbf{y}$ then we use the euclidean metric which we know is a  metric.\\
Case 2: $\textbf{x} \neq \alpha \textbf{y}$, which is really just the sum of two different metrics, which we proved in the previous exercise is a metric. 

\exercise{5.3}\\
\begin{enumerate}
    \item 
Let $X = \mathbb{R}$ and $d(\textbf{x}, \textbf{y}) = 0$. Positivity and the triangle inequality hold trivially, but here we have that $d(\textbf{x}, \textbf{y}) = 0$ when $\textbf{x} \neq \textbf{y}$.  \\
    \item
Let $X = \mathbb{R}$ and $d(\textbf{x} ,\textbf{y}) = |\textbf{x}| \cdot |\textbf{y}|$. We see that $|\textbf{x}| \cdot |\textbf{y}| \geq 0$ and $|\textbf{x}| \cdot |\textbf{y}| = |\textbf{y}| \cdot |\textbf{x}|$. But $|\textbf{x}| \cdot |\textbf{y}| \leq |\textbf{x}| \cdot |\textbf{z}| + |\textbf{z}| \cdot |\textbf{y}|$ does not hold if $\textbf{x} = 1, \textbf{y} = 2, \textbf{z} = 0$. 
\end{enumerate}

\exercise{5.4}\\
To show it is a metric, we show positive definite, symmetric, and triangle inequality. 
\begin{enumerate}
    \item Positive definite:
Because $d(\textbf{x}, \textbf{y}) \geq 0$ we know that $\rho (\textbf{x}, \textbf{y}) \geq 0$.\\
    \item Symmetry: Note,
\begin{align*}
    \rho (\textbf{x}, \textbf{y}) &= \frac{d(\textbf{x}, \textbf{y})}{1 + d(\textbf{x}, \textbf{y})} \\
    &= \frac{d(\textbf{y}, \textbf{x})}{1 + d(\textbf{y}, \textbf{x})} \\
    &=  \rho (\textbf{y}, \textbf{x})
\end{align*}

    \item Triangle inequality
Note, by the triangle inequality on $d$:
\begin{align*}
    \frac{d(x,y)}{1+d(x,y)}&\leq \frac{d(x,z)+d(z,y)}{1+d(x,z)+d(z,y)} \\
    & \leq \frac{d(x,z)}{1+d(x,z)+d(z,y)} + \frac{d(z,y)}{1+d(x,z)+d(z,y)} \\
    &  \leq \frac{d(x,z)}{1+d(x,z)} + \frac{d(z,y)}{1+d(z,y)}
\end{align*}
\end{enumerate}
Thus, it is a metric.

\exercise{5.5}\\
Note, by the triangle inequality,
\begin{align*}
    d(x,y) & \leq d(x,z)+d(z,y) \\
    d(x,z)& \geq d(x,y)-d(y,z)\\
\end{align*}
Thus, it is sufficent to show that $d(x,z) \geq d(y,z) - d(x,y)$. Note, by the triangle inequality, 
\begin{align*}
    d(y,z) \leq d(y,x) + d(x,z) \\
    d(x,z) \geq d(y,z) - d(x,y) \\
\end{align*}
Which is the desired result.

\exercise{5.6}\\
To show $f:X \rightarrow \mathbb{R}$ is continious, we must show 
$|f(x)-f(y)| < \epsilon$ if $d(x,y) < \delta$ for $x,y \in X$.
Suppose $d(x,y) < \delta$, and let $\epsilon = \delta$. Note, 
\[|f(x) - f(y)| = |d(x,D) - d(D,y) | \leq d(x,y) = \delta\]
Thus, $ |f(x) - f(y)| < \epsilon \quad \forall x \in X$, and $f$ is continious.

\exercise{5.7}\\
Note, by setting $\epsilon = \delta$ we know that $f(x) = x$ is continious, and every multivariable polynomial is a composition of sums, products, and scaler
multiplications of continious single variate functions. Thus, by Prop. 5.1.21, all multi-variable polynomials are continious.




\end{document}

