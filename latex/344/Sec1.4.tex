\documentclass[letterpaper,12pt]{article}

\usepackage{threeparttable}
\usepackage{geometry}
\geometry{letterpaper,tmargin=1in,bmargin=1in,lmargin=1.25in,rmargin=1.25in}
\usepackage[format=hang,font=normalsize,labelfont=bf]{caption}
\usepackage{amsmath}
\usepackage{multirow}
\usepackage{array}
\usepackage{delarray}
\usepackage{amssymb}
\usepackage{amsthm}
\usepackage{lscape}
\usepackage{natbib}
\usepackage{setspace}
\usepackage{float,color}
\usepackage[pdftex]{graphicx}
\usepackage{mathrsfs}  
\usepackage{pdfsync}
\usepackage{verbatim}
\usepackage{placeins} \usepackage{geometry}
\usepackage{pdflscape}
\synctex=1
\usepackage{hyperref}
\usepackage{amssymb}
\hypersetup{colorlinks,linkcolor=red,urlcolor=blue,citecolor=red}
\usepackage{bm}


\theoremstyle{definition}
\newtheorem{theorem}{Theorem}
\newtheorem{acknowledgement}[theorem]{Acknowledgement}
\newtheorem{algorithm}[theorem]{Algorithm}
\newtheorem{axiom}[theorem]{Axiom}
\newtheorem{case}[theorem]{Case}
\newtheorem{claim}[theorem]{Claim}
\newtheorem{conclusion}[theorem]{Conclusion}
\newtheorem{condition}[theorem]{Condition}
\newtheorem{conjecture}[theorem]{Conjecture}
\newtheorem{corollary}[theorem]{Corollary}
\newtheorem{criterion}[theorem]{Criterion}
\newtheorem{definition}{Definition} % Number definitions on their own
\newtheorem{derivation}{Derivation} % Number derivations on their own
\newtheorem{example}[theorem]{Example}
\newtheorem*{exercise}{Exercise} % Number exercises on their own
\newtheorem{lemma}[theorem]{Lemma}
\newtheorem{notation}[theorem]{Notation}
\newtheorem{problem}[theorem]{Problem}
\newtheorem{proposition}{Proposition} % Number propositions on their own
\newtheorem{remark}[theorem]{Remark}
\newtheorem{solution}[theorem]{Solution}
\newtheorem{summary}[theorem]{Summary}
\bibliographystyle{aer}
\newcommand\ve{\varepsilon}
\renewcommand\theenumi{\roman{enumi}}
\newcommand\norm[1]{\left\lVert#1\right\rVert}

\title{Math Sec 1.4}
\author{Rex McArthur}

\begin{document}
\maketitle
\exercise{1.20}\\
Proof: We must prove properties (v) through (viii)\\
Let $(\mathbf{x}+W) \in V/W$, $(\mathbf{y}+W) \in V/W$, $a,b \in \mathbb{F}$\\
(v): 
\begin{align*}
    a \boxdot [(\mathbf{x} + W ) \boxplus ( \mathbf{y} +W)] 
    & = a \boxdot [( \mathbf{x}+\mathbf{y})+W]\\
    & = (a \mathbf{x} + a \mathbf{y} +W)\\
    & = a \boxdot (\mathbf{x}+W) \boxplus a \boxdot (\mathbf{y}+W)
\end{align*}
(vi):
\begin{align*}
    (a+b) \boxplus ( \mathbf{x} + W) &= (a+b)\mathbf{x} + W\\
    & = (a \mathbf{x} + b \mathbf{x} + W)\\
    & = a \boxplus(\mathbf{x} + W) \boxplus b \boxdot (\mathbf{x}+W)
\end{align*}
(vii):
\begin{align*}
    1\boxdot (\mathbf{x} + W) & = (1 \mathbf{x}) + W\\
    & = (\mathbf{x} + W)
\end{align*}
(viii):
\begin{align*}
    (ab) \boxdot (\mathbf{x} + W) & = a \boxdot(b \mathbf{x} + W)\\
    & = (ab \mathbf{x}) + W \\
    & = (ba \mathbf{x} + W)\\
    & = b \boxdot (a \mathbf{x} + W)\\
    & = (ba) \boxdot (\mathbf{x} + W)
\end{align*}

\exercise{1.21}\\
\begin{align*}
    (a \boxdot (\mathbf{x} + W)) \boxplus ( b \boxdot (\mathbf{y} +W)) & = (a \mathbf{x}+W)
        \boxplus (b \mathbf{y} + W)\\
        & = (a \mathbf{x} + b \mathbf{y}) +W
\end{align*}

\exercise{1.22}\\
Proof: By definition 1.4.1, $\mathbf{x} - \mathbf{y}$ is obviously in V because both $\mathbf{x}$, and $ \mathbf{y}$ are in V. Thus they are all in the same coset, and equivalent to eachother, and that coset is the only element of the quotient space V/V.

\exercise{1.23}\\
Let $\phi$:V/{$\mathbf{0}$} $\rightarrow$ V be defined such that for each $x \in V/\{\mathbf{0}\}$ $ \phi(\mathbf{x}) = \mathbf{x}$



\end{document}

