\documentclass{article}
\usepackage{threeparttable}
\usepackage{geometry}
\geometry{letterpaper,tmargin=0.25in,bmargin=0.25in,lmargin=0.25in,rmargin=0.25in}
\usepackage[format=hang,font=normalsize,labelfont=bf]{caption}
\usepackage{amsmath}
\usepackage{anyfontsize}
\usepackage{dsfont}
\usepackage{mathrsfs}
\usepackage{multirow}
\usepackage{array}
\usepackage{delarray}
\usepackage{listings}
\usepackage{amssymb}
\usepackage{amsthm}
\usepackage{lscape}
\usepackage{natbib}
\usepackage{setspace}
\usepackage{float,color}
\usepackage[pdftex]{graphicx}
\usepackage{pdfsync}
\usepackage{verbatim}
\usepackage{placeins}
\usepackage{geometry}
\usepackage{pdflscape}
\synctex=1
\usepackage{hyperref}
\hypersetup{colorlinks,linkcolor=red,urlcolor=blue,citecolor=red}
\usepackage{bm}


\theoremstyle{definition}
\newtheorem{theorem}{Theorem}
\newtheorem{acknowledgement}[theorem]{Acknowledgement}
\newtheorem{algorithm}[theorem]{Algorithm}
\newtheorem{axiom}[theorem]{Axiom}
\newtheorem{case}[theorem]{Case}
\newtheorem{claim}[theorem]{Claim}
\newtheorem{conclusion}[theorem]{Conclusion}
\newtheorem{condition}[theorem]{Condition}
\newtheorem{conjecture}[theorem]{Conjecture}
\newtheorem{corollary}[theorem]{Corollary}
\newtheorem{criterion}[theorem]{Criterion}
\newtheorem{definition}{Definition} % Number definitions on their own
\newtheorem{derivation}{Derivation} % Number derivations on their own
\newtheorem{example}[theorem]{Example}
\newtheorem{exercise}[theorem]{Exercise}
\newtheorem{lemma}[theorem]{Lemma}
\newtheorem{notation}[theorem]{Notation}
\newtheorem{problem}[theorem]{Problem}
\newtheorem{proposition}{Proposition} % Number propositions on their own
\newtheorem{remark}[theorem]{Remark}
\newtheorem{solution}[theorem]{Solution}
\newtheorem{summary}[theorem]{Summary}
\bibliographystyle{aer}
\newcommand\ve{\varepsilon}
\renewcommand\theenumi{\roman{enumi}}
\newcommand\norm[1]{\left\lVert#1\right\rVert}

\begin{document}
$\mathbf{Conditioning}$How sensative the solution is to changes in input. Nothing to do with the algorithm.
Realtive Condition number is $k(x) = \sup_{\delta x}(\frac{\|df\|}{\|f(x)\|} \big/ \frac{\|df\|}{\|x\|})$ or $k = \frac{\|Df(x)\|}{\|f(x)\|/\|x\|}$
Bounds for a matrix in the problem $Ax = b$ where x is perturbed is $k = \|A\|\frac{\|x\|}{\|b\|} \leq \|A\|\|A^{-1}\|$ Equality holds for 2 norm.
$\mathbf{Contraction Mappings}$
\textbf{DEF Contraction Mapping}: Assume $D$ is a subset of the normed linear space $(X, \| \cdot \| )$. The funcion $f:D \rightarrow D$ is a contraction mapping if there exists $0 \leq k < 1$ such that: $\| f(\textbf{x}) - f(\textbf{y}) \| \leq k \| \textbf{x} - \textbf{y} \|, \quad \forall \textbf{x},\textbf{y} \in D$. 
\textbf{REM}: Contraction mappings are continuous and Lipschitz continuous with constant $k$. 
\textbf{THRM - Contraction Mapping Principle}: Assume $D$ is a nonempty closed subset of the Banach space $(X, \| \cdot \| )$. If $f:D \rightarrow D$ is a contraction mapping then there exists a unique fixed point $\overline{\textbf{x}}$ of $f$ in $D$. 
\textbf{REM}: The error of the method of successive approximations for the $m^{th}$ approximation is at most $\frac{k^m}{1-k} \| \textbf{x}_1 - \textbf{x}_0 \|$.
\textbf{THRM 7.1.13}: Assume $D$ is a nonempty closed subset of the Banach space $(X, \| \cdot \| )$. If $f:D \rightarrow D$ and $f^n$ is a contraction mapping, then there exists a unique fixed point $\overline{\textbf{x}} \in D$ of $f$. 
\textbf{DEF Uniform Contraction Mapping}: Assume $D$ is a nonempty subset of the normed linear space $(X, \| \cdot \| )$ and B is some arbitrary set, then the function $f:D \times B \rightarrow D$ is called a unifrom contraction mapping if there exists $0 \leq \lambda < 1$ such that: $\| f(\textbf{x}_2, \textbf{y}) - f(\textbf{x}_1, \textbf{y})  \|_X \leq \lambda \| \textbf{x}_2 - \textbf{x}_1 \|_X, \quad \forall \textbf{x}_1, \textbf{x}_2 \in D,~ \forall \textbf{y} \in B$. 
\textbf{THRM - Uniform Contraction Mapping Principle}: Assume that $(X, \| \cdot \|_X )$ and $(Y, \| \cdot \|_Y )$ are Banach spaces, $U\subset X$ and $V \subset Y$ are open, and the function $f:\overline U \times V \rightarrow \overline U$ is a uniform contraction mapping with constant $ 0 \leq \lambda < 1$. We define a function $g:V \rightarrow \overline U$ that sends each $\textbf{y} \in V$ to to the unique fixed point of the contraction $f(\cdot, \textbf{y})$. If $f \in C^k(\overline U \times V, \overline U)$, where $k$ is a nonnegative integer, then $g \in C^k(V,\overline U)$. 
\textbf{THRM 7.3.1}: Lef $f:[a,b] \rightarrow \mathbb{R}$ be a differentiable function with $f(a) < 0 < f(b)$ and $0 < m \leq f'(x) \leq M$, for all $x \in [a,b]$. Given $x_0 \in [a,b]$, the sequence $(x_s)_{s=0}^{\infty}$ defined by $x_n = x_{n-1} - \frac{f(x_{n-1})}{M}$ with $n\in \mathbb{N}$ converges to the uniqu root $\overline{x} \in [a,b]$. of the equation $f(x) = 0$. In particular, $|\overline{x} - x_n| \leq \left(1-\frac{m}{M}\right)^n \frac{|f(x_0)|}{m}$ for all $n \in \mathbb{N}$.
\textbf{THRM - Newton's Method (Scalar Version)}: Let $f:[a,b] \rightarrow \mathbb{R}$ be $C^2$. If for some $\overline{x}$, we have $f(\overline{x}) = 0$ and $f'(\overline{x}) \neq 0$, then the iterative mapping $x_n = x_{n-1} -  \frac{f(x_{n-1})}{f'(x_{n-1})}$ with $n \in \mathbb{N}$ converges to $\overline{x}$ quadratically, whenever $x_0$ is suficiently close to $\overline{x}$.
\textbf{THRM 7.3.12}: Let $(X, \| \cdot \|_X )$ be a Banach space and assume $f: X \rightarrow X$ is $C^1$ on an open neighborhood $U$ of the point $\overline{\textbf{x}}$. If $f(\overline{\textbf{x}}) = \textbf{0}$ and $Df(\overline{x}) \in \mathscr{B}(X)$ has a bounded inverse, then there exists $\delta > 0$ such that $\phi(\textbf{x}) = \textbf{x} - Df(\overline{\textbf{x}}^{-1}f(\textbf{x})$ is a contraction on $\overline{B(\overline{\textbf{x}}, \delta)}$.
\textbf{THRM - Newton's Method (Vector Version)}: Let $(X, \| \cdot \|_X )$ be a Banach space and assume $f:X \rightarrow X$ is $C^1$ in an open neighborhood $U$ of the point $\overline{\textbf{x}} \in X$. If $f(\overline{\textbf{x}}) = 0$, $Df(\overline{\textbf{x}}) \in \mathscr{B}(X)$ has a bounded inverse, and $Df(\textbf{x})$ is Lipschitz with constant $L$ on $U$, then the iterative map $\textbf{x}_n = \textbf{x}_{n-1} - Df(\textbf{x}_{n-1})^{-1}f(\textbf{x}_{n-1})$ converges quadratically to $\overline{\textbf{x}}$ whenever $\textbf{x}_0$ is suficiently close to $\overline{\textbf{x}}$. \textbf{THRM - The Implicit Function Theorem}: Assume that $(X, \| \cdot \|_X )$, $(Y, \| \cdot \|_Y )$, and $(Z, \| \cdot \|_Z )$ are Banach spaces, that $U$ and $V$ are open neighborhoods of $\textbf{x}_0 \in X$ and $\textbf{y}_0 \in Y$, respectively, and that $F: U \times V \rightarrow Z$ is a $C^k$ map for some $k \in \mathbb{N}$. If $D_2F(\textbf{x}_0, \textbf{y}_0) \in \mathscr{B}(Y,Z)$ has a bounded inverse, then there exists an open neighborhood $U_0 \times V_0 \subset U\times V$ of $(\textbf{x}_0, \textbf{y}_0)$ and a unique $C^k$ function $f:U_0 \rightarrow V_0$ such that $f(\textbf{x}_0) = \textbf{y}_0$ and $F(\textbf{x}, f(\textbf{x})) = F(\textbf{x}_0, \textbf{y}_0)$ for all $\textbf{x} \in U_0$. Moreover, $f$ satisfies $Df(\textbf{x}) = -D_2F(\textbf{x}, f(\textbf{x}))^{-1}D_1F(\textbf{x}, f(\textbf{x}))$ on $U_0$.\\
\textbf{THRM - The Inverse Function Theorem}: Assume that  $(X, \| \cdot \|_X )$ and $(Y, \| \cdot \|_Y )$ are Banach spaces, that $U$ and $V$ are open neighborhoods of $\textbf{x}_0 \in X$ and $\textbf{y}_0 \in Y$, respectively, and that $f:U \rightarrow V$ is a $C^k$ map, $k \in \mathbb{N}$, satisfying $f(\textbf{x}_0) = \textbf{y}_0$. If $Df(\textbf{x}_0) \in \mathscr{B}(X;Y)$ has a bounded inverse, then there exist open neighborhoods $U_0$ that is inverse to $f$. That is, $f(g(\textbf{y})) = \textbf{y}$ for all $\textbf{y} \in V_0$ and $g(f(\textbf{x})) = \textbf{x}$ for all $\textbf{x} \in U_0$. Moreover, for all $\textbf{x} \in U_0$, we have $Dg(\textbf{y}) = Df(g(\textbf{y}))^{-1}$. \\ 
\textbf{INTEGRATION} Measure is defined as: $\lambda(R) = \prod^n_{j=1} (b_j-a_j)$ Define integral to be $\mathscr{I}(s) = \int_{[a,b]} s = \sum x_I \lambda(R_I)$ Note $\mathscr{I}$ is a bounded linear transformation wrt $L^\infty$ and $\|\mathscr{I}\| = \lambda([b-a])$ by using linear extention Thm. we extend it to the closure of step functions in $L^\infty$
\textbf{Multivariable Banach Integral} Let $(X,|\ \cdot\|)$ be a banach, $\mathscr{I}:  S([a,b];X) \to X$ can be 
extended uniquely to a bounded linear transformation $ \overline{ \mathscr{I} }: \overline{S([a,b];X)}\to X$, 
where $\| \mathscr{I}(f) \|\leq \lambda([b-a]) \| f \|_{L^\infty}$ for all 
$f \in \overline S$. Moreover, $C \subset \overline S \subset L^\infty$
\textbf{integral prop} for $f,g \in \mathscr{R}([a,b];X)$, (i) $\|\int_{[a,b]} \|_X \leq \lambda([a,b])\sup_{t\in[a,b]} \|f\|_X$ (ii) Let $\|f\|_X$ denote the function $t \to \|f(t) \|_X$ from $[a,b]$ to $\mathbb{R}$. $\|\int_{[a,b]}\|_X \leq \int _{[a,b]} \|f\|_X$ (iii) If $\|f(t)\|_X \leq \|g(t)\|_X$ for every $t \in [a,b]$, then $\int_{[a,b]} \|f\|_X \leq \int_{[a,b]} \|g\|_X $. (iv) For any sequence $(f_k)_{k \in \mathbb{N}} \in \mathscr{R}$ which converges uniformly to f, $\lim_{k\to \infty} \int_{[a,b]} f_k = \int_{[a,b]} \lim_{k\to \infty} f_k = \int_{[a,b]} f$.
\textbf{Lebesgue} DEF: For any $ f \in \mathscr{R}$, $\|f\|_{L^1} = \int_{[a,b]} |f|$ Fun facts, $\|f\|_{L^1} \leq \lambda([a,b]) \|f\|_{L^\infty}$
\textbf{Measure Zero} A set $A \subset \mathbb{R}^n$ has measure zero if for any $\epsilon > 0$ there is a collection of n-intervals ${I_k}_{k \in K}$ indexed by a countable set K, such that $A \subset \cup_{k \in K}I_k$ and $\sum_{K \in K} \lambda(I_k) < \epsilon$. Funfacts, subsets of measure zero are measure zero, single points are measure zero, countable unions of measure zeros sets are measure zero. Almost everywhere functions differ on a set of measure zero. Almost everywhere is an equivalence relation. (i) if $f \leq g$ a.e. then $\int _{[a,b]} f \leq \int_{[a,b]} g$. (ii) $|\int_{[a,b]} f | \leq \int_{[a,b]} |f| = \|f\|_{L^1}$ (iii) the max, min and |f| functions are all integrable.
\textbf{MCT} if $(f_k)_{k \in \mathbb{N}} \subset L^1([a,b];\mathbb{R})$ is almost everywhere monotone increasing, and if there exists an M s.t. $\int_{[a,b]} f_k \leq M$ for all $k \in \mathbb{N}$, then $(f_k)$ is $L^1$ Cauchy, and there exists an $f \in L^1$ such that (i) $ f = \lim_{k \to \infty} f_k$ a.e. and (ii) $ \int_{[a,b]} f = \int_{[a,b]}lim_{k \to \infty} f_k = lim_{k \to \infty} \int_{[a,b]} f_k$. Same conclusion holds for decreasing and the integral is bounded below by M.
$\liminf_{k \to \infty} x_k = \lim_{k \to \infty}(\inf_{m \leq k}x_m)$
\textbf{Fatou's} Let $(f_k)$ be a sequence of integrable functions on $[a,b]$ that are almsot everywhere non-negative. If $\liminf_{k \to \infty} \int _{[a,b]} f_k < \infty$ then (i) $(\liminf_{k \to \infty}) \in L^1$ and (ii) $\int_{[a,b]} \liminf_{k \to \infty} f_k \leq \liminf_{k \to \infty} \int _{[a,b]} f_k$.
\textbf{Dominated convergence} If $(f_k) \subset L^1$ is a sequence of integrable functions that converges pointwise to f, and if there exists an 
integrable $g\in L^1$ s.t. $|f_k| \leq g$ a.e. then $f \in L^1$ and $\int_{[a,b]} \lim_{k \to \infty} = \int_{[a,b]} f = \lim_{k \to \infty} \int_{[a,b]} f_k$
Funfact, if f is in $L^1$ and $\sum^\infty_{k=0}\int_{[a,b]}|f_k| < \infty$ then, $\sum^\infty_{k=0}\int_{[a,b]}f_k = \int_{[a,b]}\sum^\infty_{k=0}f_k$
\textbf{Fubini}
\textbf{Leibniz} Let $ X = (r,s) \subset \mathbb{R}$ be an open interval, let $T = [a,b] \subset \mathbb{R}$ be a compact interval, and let $f:X \times T \to \mathbb{R}$ be a continous function. If f has continuous partial derivatives $\frac{\delta f(x,t)}{\delta x}$ at each point $(x,t) \in X \times T$, then the function $\phi(x) = \int_a^b f(x,t) dt$ is differentiable at each point $x \in X$ and $\frac{d}{dx}\phi(x) = \int ^b_a \frac{\delta f(x,t)}{\delta x}dt$
\textbf{Diffeomorphism} Let U and V be open subsets of $\mathbb{R}^n$, we say that $f:U \to V$ is a diffeomorphism if f is a $C^1$ bijection such that $f^{-1}$ is also $C^1$
\textbf{Change of Variable} $\int_Y f dy = \int _X (f$




\end{document}
