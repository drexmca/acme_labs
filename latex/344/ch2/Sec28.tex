\documentclass[letterpaper,12pt]{article}

\usepackage{threeparttable}
\usepackage{geometry}
\geometry{letterpaper,tmargin=1in,bmargin=1in,lmargin=1.25in,rmargin=1.25in}
\usepackage[format=hang,font=normalsize,labelfont=bf]{caption}
\usepackage{amsmath}
\usepackage{multirow}
\usepackage{array}
\usepackage{delarray}
\usepackage{amssymb}
\usepackage{amsthm}
\usepackage{lscape}
\usepackage{natbib}
\usepackage{setspace}
\usepackage{float,color}
\usepackage[pdftex]{graphicx}
\usepackage{mathrsfs}  
\usepackage{pdfsync}
\usepackage{verbatim}
\usepackage{placeins} \usepackage{geometry}
\usepackage{pdflscape}
\synctex=1
\usepackage{hyperref}
\hypersetup{colorlinks,linkcolor=red,urlcolor=blue,citecolor=red}
\usepackage{bm}
\usepackage{amssymb}


\theoremstyle{definition}
\newtheorem{theorem}{Theorem}
\newtheorem{acknowledgement}[theorem]{Acknowledgement}
\newtheorem{algorithm}[theorem]{Algorithm}
\newtheorem{axiom}[theorem]{Axiom}
\newtheorem{case}[theorem]{Case}
\newtheorem{claim}[theorem]{Claim}
\newtheorem{conclusion}[theorem]{Conclusion}
\newtheorem{condition}[theorem]{Condition}
\newtheorem{conjecture}[theorem]{Conjecture}
\newtheorem{corollary}[theorem]{Corollary}
\newtheorem{criterion}[theorem]{Criterion}
\newtheorem{definition}{Definition} % Number definitions on their own
\newtheorem{derivation}{Derivation} % Number derivations on their own
\newtheorem{example}[theorem]{Example}
\newtheorem*{exercise}{Exercise} % Number exercises on their own
\newtheorem{lemma}[theorem]{Lemma}
\newtheorem{notation}[theorem]{Notation}
\newtheorem{problem}[theorem]{Problem}
\newtheorem{proposition}{Proposition} % Number propositions on their own
\newtheorem{remark}[theorem]{Remark}
\newtheorem{solution}[theorem]{Solution}
\newtheorem{summary}[theorem]{Summary}
\bibliographystyle{aer}
\newcommand\ve{\varepsilon}
\renewcommand\theenumi{\roman{enumi}}

\title{Math Sec 1.6}
\author{Rex McArthur\\Math 320}


\begin{document}
\maketitle

\exercise{2.39}
The 7 inversions are
\[ (4,3), (4,2),(3,2),(6,5),(9,8),(9,7),(8,7)\]

\exercise{2.40}
By Thm. 2.7.22, 
\[ \text{det} (A) = \text{det} (A^T) \implies \text{det}(\bar A) = \text{det} (\bar A^T) = \text{det} (A^H)\]
Thus, it is sufficent to show that $\text{det} (\bar A) = \overline{\text{det} (A)}$
\[\text{det}(\overline{A}) = \sum^{}_{\sigma \in S_n} \text{sign} (\sigma) \overline{a_{1 \sigma(1)}}~\overline{a_{2 \sigma(2)}}\cdots\overline{a_{n \sigma(n)}}\]
Note, $\text{sign} (\sigma) \in \mathbb{R}$, thus the conjugate is equal to itself.
\[\text{det}(\overline{A}) = \sum^{}_{\sigma \in S_n} \text{sign} (\sigma) \overline{a_{1 \sigma(1)}}~\overline{a_{2 \sigma(2)}}\cdots\overline{a_{n \sigma(n)}} =
\sum^{}_{\sigma \in S_n} \overline{\text{sign} (\sigma) }\overline{a_{1 \sigma(1)}}~\overline{a_{2 \sigma(2)}}\cdots\overline{a_{n \sigma(n)}} = \overline{\text{det}(A)}\]
\[\implies\overline{\text{det}(A)}= \text{det}(\bar A) = \text{det} (\bar A^T) = \text{det} (A^H)\]

\exercise{2.41}
\[ (1,2,3,4),(1,2,4,3),(1,3,4,2),(1,3,2,4),(1,4,3,2),(1,4,2,3)\]
\[ (2,1,3,4),(2,1,4,3),(2,3,1,4),(2,3,4,1),(2,4,3,1),(2,4,1,3)\]
\[ (3,1,2,4),(3,1,4,2),(3,2,1,4),(3,2,4,1),(3,4,2,1),(3,4,1,2)\]
\[ (4,1,2,3),(4,1,3,2),(4,2,1,3),(4,2,3,1),(4,3,2,1),(4,3,1,2)\]

\exercise{2.42}
\begin{align*}
    \text{det}(A) &= a_{11}a_{22}a_{33}a_{44} + \\
    & -a_{11}a_{22}a_{34}a_{43} + \\
    & a_{11}a_{23}a_{34}a_{42} + \\
    & -a_{11}a_{23}a_{32}a_{44} + \\
    & a_{11}a_{24}a_{32}a_{43} + \\
    & -a_{11}a_{24}a_{33}a_{42} + \\
    \\
    & a_{12}a_{21}a_{34}a_{43} + \\
    & -a_{12}a_{21}a_{33}a_{44} + \\
    & -a_{12}a_{23}a_{34}a_{41} + \\
    & a_{12}a_{23}a_{31}a_{44} + \\
    & -a_{12}a_{24}a_{33}a_{41} + \\
    & a_{12}a_{24}a_{31}a_{43} + \\
    \\
    & a_{13}a_{21}a_{32}a_{44} + \\
    & -a_{13}a_{21}a_{34}a_{42} + \\
    & -a_{13}a_{22}a_{31}a_{44} + \\
    & a_{13}a_{22}a_{34}a_{41} + \\
    & -a_{13}a_{24}a_{32}a_{41} + \\
    & a_{13}a_{24}a_{31}a_{42} + \\
    \\
    & -a_{14}a_{21}a_{32}a_{43} + \\
    & a_{14}a_{21}a_{33}a_{42} + \\
    & a_{14}a_{22}a_{31}a_{43} + \\
    & -a_{14}a_{22}a_{33}a_{41} + \\
    & a_{14}a_{23}a_{32}a_{41} + \\
    & -a_{14}a_{23}a_{31}a_{42} + \\
    \\
    \end{align*}
    \begin{align*}
    &= 0 \cdot 0 \cdot 7 \cdot 0 + \\
    & -0 \cdot 0 \cdot 1 \cdot 9 + \\
    & 0 \cdot 0 \cdot 1 \cdot 0 + \\
    & -0 \cdot 0 \cdot 6 \cdot 0 + \\
    & 0 \cdot 5 \cdot 6 \cdot 9 + \\
    & -0 \cdot 5 \cdot 7 \cdot 0 + \\
    \\
    & 2 \cdot 4 \cdot 1 \cdot 9 + \\
    & -2 \cdot 4 \cdot 7 \cdot 0 + \\
    & -2 \cdot 0 \cdot 1 \cdot 8 + \\
    & 2 \cdot 0 \cdot 0 \cdot 0 + \\
    & 2 \cdot 5 \cdot 7 \cdot 8 + \\
    & 2 \cdot 5 \cdot 0 \cdot 9 + \\
    \\
    & 3 \cdot 4 \cdot 6 \cdot 0 + \\
    & -3 \cdot 4 \cdot 1 \cdot 0 + \\
    & -3 \cdot 0 \cdot 0 \cdot 0 + \\
    & 3 \cdot 0 \cdot 1 \cdot 8 + \\
    & -3 \cdot 5 \cdot 6 \cdot 8 + \\
    & 3 \cdot 5 \cdot 0 \cdot 0 + \\
    \\
    & -0 \cdot 4 \cdot 6 \cdot 9 + \\
    & 0 \cdot 4 \cdot 7 \cdot 0 + \\
    & 0 \cdot 0 \cdot 0 \cdot 9 + \\
    & -0 \cdot 0 \cdot 7 \cdot 8 + \\
    & 0 \cdot 0 \cdot 6 \cdot 8 + \\
    & -0 \cdot 0 \cdot 0 \cdot 0 + \\
    \\
    & = -88
\end{align*}


\exercise{2.43}
Proof: Suppose $A_{ij} = 0~\forall i$\\
\[\text{det}(A) = \sum^{}_{\sigma \in S_n} \text{sign} (\sigma) a_{1 \sigma (1)}a_{2 \sigma (2)}\cdots a_{n \sigma (n)}\]
Thus, for everysingle elementary product with i, it will equal zero. And since the determinent is equal to the summation of all elemantary products, the sum of j zeros is zeros. Supposing for a column of zeros is identical.
Therefore,
\[\text{det}(A) = \sum^{}_{\sigma \in S_n} \text{sign} (\sigma) 0 \cdot 0 \cdots \cdot 0 = 0\]



\end{document}

