\documentclass[letterpaper,12pt]{article}

\usepackage{threeparttable}
\usepackage{geometry}
\geometry{letterpaper,tmargin=1in,bmargin=1in,lmargin=1.25in,rmargin=1.25in}
\usepackage[format=hang,font=normalsize,labelfont=bf]{caption}
\usepackage{amsmath}
\usepackage{mathrsfs}
\usepackage{multirow}
\usepackage{array}
\usepackage{delarray}
\usepackage{listings}
\usepackage{amssymb}
\usepackage{amsthm}
\usepackage{lscape}
\usepackage{natbib}
\usepackage{setspace}
\usepackage{float,color}
\usepackage[pdftex]{graphicx}
\usepackage{pdfsync}
\usepackage{verbatim}
\usepackage{placeins}
\usepackage{geometry}
\usepackage{pdflscape}
\synctex=1
\usepackage{hyperref}
\hypersetup{colorlinks,linkcolor=red,urlcolor=blue,citecolor=red}
\usepackage{bm}


\theoremstyle{definition}
\newtheorem{theorem}{Theorem}
\newtheorem{acknowledgement}[theorem]{Acknowledgement}
\newtheorem{algorithm}[theorem]{Algorithm}
\newtheorem{axiom}[theorem]{Axiom}
\newtheorem{case}[theorem]{Case}
\newtheorem{claim}[theorem]{Claim}
\newtheorem{conclusion}[theorem]{Conclusion}
\newtheorem{condition}[theorem]{Condition}
\newtheorem{conjecture}[theorem]{Conjecture}
\newtheorem{corollary}[theorem]{Corollary}
\newtheorem{criterion}[theorem]{Criterion}
\newtheorem{definition}{Definition} % Number definitions on their own
\newtheorem{derivation}{Derivation} % Number derivations on their own
\newtheorem{example}[theorem]{Example}
\newtheorem{exercise}[theorem]{Exercise}
\newtheorem{lemma}[theorem]{Lemma}
\newtheorem{notation}[theorem]{Notation}
\newtheorem{problem}[theorem]{Problem}
\newtheorem{proposition}{Proposition} % Number propositions on their own
\newtheorem{remark}[theorem]{Remark}
\newtheorem{solution}[theorem]{Solution}
\newtheorem{summary}[theorem]{Summary}
\bibliographystyle{aer}
\newcommand\ve{\varepsilon}
\renewcommand\theenumi{\roman{enumi}}
\newcommand\norm[1]{\left\lVert#1\right\rVert}

\begin{document}

\title{Math 320 Homework 2.1}
\author{Chris Rytting}
\maketitle

\subsection*{2.1}
Since Abe said we don't actually have to list out the subgraphs, as the question indicates, I will show how to derive the number of subgraphs.
Let $G_i = (V_i, E_i)$ be a sequence of subgraphs of $G$ for $i = 1,2,\cdots,48$, as we know the following:
\\
For subgraphs with one vertex, there will be 4 subgraphs, as there are no edges between a vertex and itself. 
\\For subgraphs with two vertices, there will be 10 subgraphs, since between $a$ and $d$, there is a subgraph with no edge but that includes both vertices and a subgraph with an edge that includes both vertices. On the other hand, there exists a subgraph between $b$ and $d$, there is only one subgraph since there is no edge in $G$ between $b$ and $d$. 
\\For subgraphs with three vertices, there will be 18 total subgraphs.
\\For subgraphs with four vertices, there will be 16 total subgraphs.
\\Therefore, we have that there are 
\[4+ 10 + 16+ 18 = 48\]
total subgraphs.

\subsection*{2.2}

For undirected graphs, we know there are $\binom 72 = 21$ edges between the seven vertices. Therefore, the number of distinct graphs is given by $\binom {21}{13} = 203490$. \\\\
For directed graphs, we know there are $2\binom 72 = 42$ edges between the seven vertices, (since now there are simply twice as many). Therefore, the number of distinct graphs is given by $\binom {42}{13} = 25518731280$. \\\\

\subsection*{2.3}

Let $A$ be our adjacency matrix. Then we have that 
\[A = 
\begin{bmatrix}
    0&0&1&0&0&0\\
    1&0&0&0&0&0\\
    1&0&0&0&1&1\\
    0&1&1&0&0&0\\
    0&0&0&0&0&1\\
    0&0&0&1&0&0\\
\end{bmatrix}
\]

\[A^4 = 
\begin{bmatrix}
    0&0&0&3&0&1\\
    2&0&0&0&1&1\\
    4&1&1&0&2&2\\
    0&3&3&1&0&0\\
    2&0&0&0&1&1\\
    0&0&0&3&0&1\\
\end{bmatrix}
\]
$\implies$ by Proposition 2.1.13, we have that there are 3 length-4 paths from node 1 to 4.

\subsection*{2.4}

Let $A$ be our adjacency matrix. Then we have that 
\[A = 
\begin{bmatrix}
    0&1&1&0&0&0\\
    1&0&1&0&1&0\\
    1&1&0&0&1&1\\
    0&0&0&0&0&0\\
    0&1&1&0&0&1\\
    0&0&1&0&1&0\\
\end{bmatrix}
\]

\[A^3 = 
\begin{bmatrix}
    2&5&6&0&3&3\\
    5&4&7&0&7&3\\
    6&7&6&0&7&6\\
    0&0&0&0&0&0\\
    3&7&7&0&4&5\\
    3&3&6&0&5&2\\
\end{bmatrix}
\]
$\implies$ by Proposition 2.1.13, we have that there are 6 length-3 paths from node 3 to 3.

\subsection*{2}









\end{document}
