\documentclass[letterpaper,12pt]{article}

\usepackage{threeparttable}
\usepackage{geometry}
\geometry{letterpaper,tmargin=1in,bmargin=1in,lmargin=1.25in,rmargin=1.25in}
\usepackage[format=hang,font=normalsize,labelfont=bf]{caption}
\usepackage{amsmath}
\usepackage{multirow}
\usepackage{array}
\usepackage{delarray}
\usepackage{amssymb}
\usepackage{amsthm}
\usepackage{lscape}
\usepackage{natbib}
\usepackage{setspace}
\usepackage{float,color}
\usepackage[pdftex]{graphicx}
\usepackage{mathrsfs}  
\usepackage{pdfsync}
\usepackage{verbatim}
\usepackage{placeins} \usepackage{geometry}
\usepackage{pdflscape}
\synctex=1
\usepackage{hyperref}
\hypersetup{colorlinks,linkcolor=red,urlcolor=blue,citecolor=red}
\usepackage{bm}


\theoremstyle{definition}
\newtheorem{theorem}{Theorem}
\newtheorem{acknowledgement}[theorem]{Acknowledgement}
\newtheorem{algorithm}[theorem]{Algorithm}
\newtheorem{axiom}[theorem]{Axiom}
\newtheorem{case}[theorem]{Case}
\newtheorem{claim}[theorem]{Claim}
\newtheorem{conclusion}[theorem]{Conclusion}
\newtheorem{condition}[theorem]{Condition}
\newtheorem{conjecture}[theorem]{Conjecture}
\newtheorem{corollary}[theorem]{Corollary}
\newtheorem{criterion}[theorem]{Criterion}
\newtheorem{definition}{Definition} % Number definitions on their own
\newtheorem{derivation}{Derivation} % Number derivations on their own
\newtheorem{example}[theorem]{Example}
\newtheorem*{exercise}{Exercise} % Number exercises on their own
\newtheorem{lemma}[theorem]{Lemma}
\newtheorem{notation}[theorem]{Notation}
\newtheorem{problem}[theorem]{Problem}
\newtheorem{proposition}{Proposition} % Number propositions on their own
\newtheorem{remark}[theorem]{Remark}
\newtheorem{solution}[theorem]{Solution}
\newtheorem{summary}[theorem]{Summary}
\bibliographystyle{aer}
\newcommand\ve{\varepsilon}
\renewcommand\theenumi{\roman{enumi}}
\newcommand\norm[1]{\left\lVert#1\right\rVert}

\title{Math Sec 1.4}
\author{Rex McArthur}

\begin{document}
\maketitle
\exercise{1.19}\\
(i)
Proof: Given an integer $a = \sum^{n-1}_{k=0} a_k10^k = a_010^0+\dots+
    a_{n-1}10^{n-1}$, Note that\\
        \begin{align*}
            a &= \sum{a_k10^k}\\
            & = a_0+\dots+a_{n-1}10^{n-1}\\
            & = (9a_1 + 99a_2 + \dots +(10^{n-1}-1)a_{n-1}) + (a_0 + a_1 + \dots
                + a_{n-1}\\
            & = 3(3a_1 + 33 a_2 + \dots +  (10^{n-1} -1)/3 a_{n-1}) + 
            (a_0+\dots+a_{n-1})\\
            & = 3(3a_1 + 33 a_2 + \dots +  (10^{n-1} -1)/3 a_{n-1}) + 
            \sum^{n-1}_{k=0} a_k
        \end{align*}
    Note that $3|a$ if and only if $3|\sum^{n-1}_{k=0} a_k$, by definition of 
    divisibility, because it is apparent that 3 divides the first term.\\
(ii)
Proof: Given an integer $a = \sum^{n-1}_{k=0} a_k10^k = a_010^0+\dots+
    a_{n-1}10^{n-1}$, Note that\\
        \begin{align*}
            a &= \sum{a_k10^k}\\
            & = a_0+\dots+a_{n-1}10^{n-1}\\
            & = (9a_1 + 99a_2 + \dots +(10^{n-1}-1)a_{n-1}) + (a_0 + a_1 + \dots
                + a_{n-1})\\
            & = 9(1a_1 + 11 a_2 + \dots +  (10^{n-1} -1)/3 a_{n-1}) + 
            (a_0+\dots+a_{n-1})\\
            & = 9(1a_1 + 11 a_2 + \dots +  (10^{n-1} -1)/3 a_{n-1}) + 
            \sum^{n-1}_{k=0} a_k
        \end{align*}
    Note that $9|a$ if and only if $9|\sum^{n-1}_{k=0} a_k$, by definition of 
    divisibility, because it is apparent that 9 divides the first term.\\
(iii)
Proof: Given an integer $a = \sum^{n-1}_{k=0} a_k10^k = a_010^0+\dots+
    a_{n-1}10^{n-1}$, Note that\\
\begin{align*}
    a &= \sum^{n-1}_{k = 0} a_k 10^k \\
    & = a_0 10^0 + a_1 + 10^1 + \dots + a_{n-1} 10^{n-1}\\
    & = (11a_1 + 99a_2 + 1001a_3 + \dots + (10^{n-1}-1)+a_{n-1}) + (a_0 - a_1 + 
        a_2 - \dots + (-1)^{n-1} a_{n-1})\\
   &(\text{Where the coefficients of $a_i^{th}$ term given by $10^{i}-(-1)^{i-1}$})\\
   & = 11(a_1 + 9a_2+91a_3+909a_4\dots+( (10^{n-1}-1)/11)a_{n-1})+(a_0-a_1+a_2\dots
        +(-1)^{n-1}a_{n-1})
\end{align*}
Note that $a$ is divisible by 11 if and only if the second expression is divisible
by 11, by definition of divisiblity, since the entire first part is divisible by 11.

\exercise{1.20}\\
Since $a \equiv b(\text{mod }c)$, we know that $a-b = cn$ for some $n \in \mathbb{Z}$.
Also, since $d|c$, we know that $c = dm$ for some $m \in \mathbb{Z}$. Note,\\
\[a-b = cn = dmn \implies d|(a-b) \implies a \equiv b(\text{mod }d)\]
where $k = mn \implies k \in \mathbb{Z}$.

\exercise{1.21}\\
Note that $4^2 = 16 \equiv_{12} 4$.
\[
    34^{34} \equiv_{12} -2^{34} \equiv_{12} 4^{17} \equiv_{12} 4^2 4^2 4^2 4^2 4^2
    4^2 4^2 4^2 4 \equiv_{12} 4^8 \equiv_{12} 4^2 4^2 4^2 \equiv_{12} 4^3 \equiv_{12} 4
\]

\exercise{1.22}\\
(i): By Fermat's Little Theorem, \\
\[
    14^{127} 14 \equiv_{127} 14^2 \equiv_{127} 69
\]
(ii): By Fermat's Little Theorem, \\
\[
    (18^2)^{127} \equiv_{127} 18^2 \equiv_{127} 324 \equiv_{127} 197 \equiv_{127} 70
\]
(iii): By Fermat's Little Theorem, \\
\[
    (25^5)^{127} 25^5 \equiv_{127} 625^5 \equiv_{127} -5^5 \equiv_{127} -625*5 
    \equiv_{127} -5*-5 \equiv_{127} 25
\]

\exercise{1.23}\\
The necessary and sufficent conditions are that gcd$(x,c) = 1$.\\
Proof: Suppose gcd$(x,c) = 1$, and $ax \equiv bx (\text{mod }c)$. Thus $c|(ax - bx)$,
and $c|(a-b)x$. Because gcd$(x,c) = 1$, $c|(a-b)$ by Proposition 1.4.9, and 
$a \equiv b (\text{mod }c)$.\\
Now, suppose $ax \equiv bx (\text{mod } c) \rightarrow a \equiv b (\text{mod } c)$. Thus,
$c|(a-b)x$ implies that $c|(a-b)$. This will only happen if c and x are relatively prime,
which means that gcd$(c,x) = 1$.

\exercise{1.24}\\
See attached code.

\exercise{1.25}\\
Note that the Eucledian algorithm has at most a steps, because if $a = b$, it has one step, and the worst case scenario is it decreases by one each time, and the remainder drops by one each time. Since the temporal complexity of the division algorithm is $O(f(a))$, it follows that the Eucledian algorithim would have a temporal complexity of at the very omost $O(af(a))$.

\end{document}

