\documentclass[letterpaper,12pt]{article}

\usepackage{threeparttable}
\usepackage{geometry}
\geometry{letterpaper,tmargin=1in,bmargin=1in,lmargin=1.25in,rmargin=1.25in}
\usepackage[format=hang,font=normalsize,labelfont=bf]{caption}
\usepackage{amsmath}
\usepackage{multirow}
\usepackage{array}
\usepackage{delarray}
\usepackage{amssymb}
\usepackage{amsthm}
\usepackage{lscape}
\usepackage{natbib}
\usepackage{setspace}
\usepackage{float,color}
\usepackage[pdftex]{graphicx}
\usepackage{mathrsfs}  
\usepackage{pdfsync}
\usepackage{verbatim}
\usepackage{placeins} \usepackage{geometry}
\usepackage{pdflscape}
\synctex=1
\usepackage{hyperref}
\hypersetup{colorlinks,linkcolor=red,urlcolor=blue,citecolor=red}
\usepackage{bm}
\usepackage{amssymb}


\theoremstyle{definition}
\newtheorem{theorem}{Theorem}
\newtheorem{acknowledgement}[theorem]{Acknowledgement}
\newtheorem{algorithm}[theorem]{Algorithm}
\newtheorem{axiom}[theorem]{Axiom}
\newtheorem{case}[theorem]{Case}
\newtheorem{claim}[theorem]{Claim}
\newtheorem{conclusion}[theorem]{Conclusion}
\newtheorem{condition}[theorem]{Condition}
\newtheorem{conjecture}[theorem]{Conjecture}
\newtheorem{corollary}[theorem]{Corollary}
\newtheorem{criterion}[theorem]{Criterion}
\newtheorem{definition}{Definition} % Number definitions on their own
\newtheorem{derivation}{Derivation} % Number derivations on their own
\newtheorem{example}[theorem]{Example}
\newtheorem*{exercise}{Exercise} % Number exercises on their own
\newtheorem{lemma}[theorem]{Lemma}
\newtheorem{notation}[theorem]{Notation}
\newtheorem{problem}[theorem]{Problem}
\newtheorem{proposition}{Proposition} % Number propositions on their own
\newtheorem{remark}[theorem]{Remark}
\newtheorem{solution}[theorem]{Solution}
\newtheorem{summary}[theorem]{Summary}
\bibliographystyle{aer}
\newcommand\ve{\varepsilon}
\renewcommand\theenumi{\roman{enumi}}

\title{Math Sec 1.7}
\author{Rex McArthur\\Math 320}


\begin{document}
\maketitle
\exercise{1.37}\\
Proof: We proceed by induction. For n=1, there are 1 permutations possible, 
obviously. 1 element can only be arranged in 1 unique way.\\
Assume for n-1 elements, there are $(n-1)!$ permutations. If you add one
element to the set, you have n times as many permutations, one for each 
spot to put the last $(n^{th})$ element. Thus there are $n(n-1)! = n!$ 
permutations.

\exercise{1.38}\\
\begin{enumerate}
    \item $6!$
    \item $5!\cdot2$
    \item $4!\cdot3!$
    \item $2(3!)^2$ because you have two groups, with three people with 
        three places to sit.
\end{enumerate}

\exercise{1.39}\\
$C(4,2)^2 = 6^2$ 6 ways, for each pair. \\
$C(13,2) = 78$ Ways to pick the different ranks of pairs\\
$4 \cdot C(11,1) = 44$ Picking the rank of the last card, and the suit\\
Thus, $6^2\cdot 11\cdot 4 = 123,552$

\exercise{1.40}\\
Given they chose 5 balls, and you have to pick the right 'powerball', 
the number of ways to match 3 of the 5 are $C(5,3) = 10$, 
and the other two have to come from the other 54, so $C(54,2) = 1,431$. 
Thus, 14,310 total combanations win \$100. \\
We were given that the total number of combanations was 175,223,510. Thus,
the probability is $\frac{14,310}{175,223,510} \approx .000081667$

\exercise{1.41}\\
(i):
\begin{align*}
    S_n & = \sum^n _{k=1} \binom nk k \\
    & =  \sum^n _{k=1} \binom n {(n-k)} (n-k) \text{    
        By symmetry of Binomial Thm.}\\
    & =  n\sum^n _{k=1} \binom n {(n-k)} - \sum^n _{k=1} \binom n {(n-k)} k \\
    & =  n\sum^n _{k=1} \binom n {n} - \sum^n _{k=1} \binom n {n} k \\
    S_n & = n2^n - S_n \\
    ~2 S_n & = n2^n \\
    S_n & = n2^{n-1}
\end{align*}
(ii):
%TODO Finsish this piece here

\exercise{1.42}\\





\end{document}

