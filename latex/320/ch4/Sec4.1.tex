\documentclass[letterpaper,12pt]{article}

\usepackage{threeparttable}
\usepackage{geometry}
\geometry{letterpaper,tmargin=1in,bmargin=1in,lmargin=1.25in,rmargin=1.25in}
\usepackage[format=hang,font=normalsize,labelfont=bf]{caption}
\usepackage{amsmath}
\usepackage{multirow}
\usepackage{array}
\usepackage{delarray}
\usepackage{amssymb}
\usepackage{amsthm}
\usepackage{listings}
\usepackage{lscape}
\usepackage{natbib}
\usepackage{setspace}
\usepackage{float,color}
\usepackage[pdftex]{graphicx}
\usepackage{mathrsfs}  
\usepackage{pdfsync}
\usepackage{verbatim}
\usepackage{placeins} \usepackage{geometry}
\usepackage{pdflscape}
\synctex=1
\usepackage{hyperref}
\hypersetup{colorlinks,linkcolor=red,urlcolor=blue,citecolor=red}
\usepackage{bm}
\usepackage{amssymb}


\theoremstyle{definition}
\newtheorem{theorem}{Theorem}
\newtheorem{acknowledgement}[theorem]{Acknowledgement}
\newtheorem{algorithm}[theorem]{Algorithm}
\newtheorem{axiom}[theorem]{Axiom}
\newtheorem{case}[theorem]{Case}
\newtheorem{claim}[theorem]{Claim}
\newtheorem{conclusion}[theorem]{Conclusion}
\newtheorem{condition}[theorem]{Condition}
\newtheorem{conjecture}[theorem]{Conjecture}
\newtheorem{corollary}[theorem]{Corollary}
\newtheorem{criterion}[theorem]{Criterion}
\newtheorem{definition}{Definition} % Number definitions on their own
\newtheorem{derivation}{Derivation} % Number derivations on their own
\newtheorem{example}[theorem]{Example}
\newtheorem*{exercise}{Exercise} % Number exercises on their own
\newtheorem{lemma}[theorem]{Lemma}
\newtheorem{notation}[theorem]{Notation}
\newtheorem{problem}[theorem]{Problem}
\newtheorem{proposition}{Proposition} % Number propositions on their own
\newtheorem{remark}[theorem]{Remark}
\newtheorem{solution}[theorem]{Solution}
\newtheorem{summary}[theorem]{Summary}
\bibliographystyle{aer}
\newcommand\ve{\varepsilon}
\renewcommand\theenumi{\roman{enumi}}

\title{Sec 4.1 \\Math 320}
\author{Rex McArthur}

\begin{document}
\maketitle

\exercise{4.1}\\

By definition, we know that we can rewrite the function as,
\begin{align*}
f(x) &=  c' + \sum^{\infty}_{k=1} a_k \text{cos}(kx) + b_k \text{sin} (kx)
\end{align*}
Where, $c'$ is some constant.
If $f(x) + c$ is an odd function, we know that the $a_k$s are going to be zero. This is because no odd function can be the sum of odd and even functions.

Similarly, if $f(x) + c$ is an even function, $b_k$s are going to go to zero. This is because an even function is only the sum of even functions.

\exercise{4.2}\\
Note that, 
\begin{align*}
    c_k &= \frac{1}{2\pi} \left( \int^{\pi}_{-\pi} f(x)e^{-ikx}  dx \right)\\
    &= \frac{1}{2\pi} \left( \int^{0}_{-\pi} f(x)e^{-ikx}  dx + \int^{\pi}_{0} f(x)e^{-ikx}  dx \right)\\
    &= \frac{1}{2\pi} \left( \int^{0}_{-\pi} -x e^{-ikx}  dx + \int^{\pi}_{0} x e^{-ikx}  dx \right)\\
\end{align*}
\begin{align*}
    &= \frac{\frac{-1+e^{i \pi k} (1-i \pi k)}{k^2}+\frac{-1+e^{-i \pi k} (1+i \pi k)}{k^2}}{2 \pi}\\
    &= \frac{-(1 - e^{i \pi k})}{\pi k^2}
\end{align*}
and thus, $c_0 = \frac{\pi}{2}$. 

\exercise{4.3}\\
We have that
\begin{align*}
    c_k &= \frac{1}{2\pi} \left( \int^{2\pi}_{0} f(x)e^{-ikx}  dx \right)\\
    &= \frac{1}{2\pi} \left( \int^{\pi}_{0} f(x)e^{-ikx}  dx + \int^{2\pi}_{\pi} f(x)e^{-ikx}  dx  \right)\\
    &= \frac{1}{2\pi} \left( \int^{\pi}_{0} -e^{-ikx}  dx + \int^{2\pi}_{\pi} e^{-ikx}  dx  \right)\\
    &= \frac{-1}{2 \pi k} (2ie^{-ik \pi} - ie^{-2ik \pi} - i)
\end{align*}
\begin{align*}
    ||f||^2 &= \langle f, f \rangle \\
    &= \left( \frac{1}{T} \int^{T}_{0} \overline {f(x)} f(x) dx \right)  \\
    &= \left( \frac{1}{2 \pi} \int^{\pi}_{0} (-1)^2 dx + \int^{2\pi}_{\pi} (1)^2 dx\right)  \\
    &= \left( \frac{1}{2 \pi} \int^{2\pi}_{0}dx\right)  \\
    &= \left( \frac{1}{2 \pi} 2\pi\right)  \\
    &= 1 
\end{align*}

\exercise{4.4}\\
By integrating by parts,
\begin{align*}
b_k &= \frac{1}{\pi} \int^{2\pi}_{0} (\pi - x) \text{sin}(kx) dx\\
&= \frac{1}{\pi} \left(  \int^{2\pi}_{0} \pi \text{sin}(kx) dx - \int^{2\pi}_{0} x \text{sin}(kx) dx\right)\\
&= \frac{1}{\pi} \left(  0 - \int^{2\pi}_{0} x \text{sin}(kx) dx\right)\\
&= -\frac{1}{\pi} \left(  \int^{2\pi}_{0} x \text{sin}(kx) dx\right)\\
&= - \frac{1}{\pi} \left(  -\frac{2\pi}{k}\right)\\
&=  \frac{2\pi}{k\pi} \\
&=  \frac{2}{k} \\
\end{align*}
and,
\begin{align*}
a_0 = \frac{1}{2\pi} \int^{2  \pi}_{0} \frac{\pi - x}{\sqrt 2} dx = 0
\end{align*}

\exercise{4.5}\\
\begin{enumerate}
    \item 
        TODO: import from Gibbs.py
    \item
\begin{align*}
    g_n(x) &= (\pi - x) - \sum^{n}_{k=1} 2 \text{sin}(kx) \\
    \implies g_n'(x) &= -1 + \sum^{n}_{k=1} 2 k \text{cos}(kx) \\
\end{align*}
We want to show that 
    \[\frac{\text{sin} \left( (2n+1)\frac{x}{2} \right)}{\text{sin} (\frac{x}{2}) }  =-1 + \sum^{n}_{k=1} 2 \text{cos}(x)\] Note that this is true if and only if 
\begin{align*}
     \sum^{n}_{k=1} 2 \text{cos}(x) & = \frac{\text{sin} \left( (2n+1) \right)\frac{x}{2} }{\text{sin} (\frac{x}{2}) } +1 \\
     \implies \sum^{n}_{k=1} 2 \text{cos}(x) \text{sin}(\frac{x}{2}) & = \text{sin}( (2n+1) \frac{x}{2}) + \text{sin}(\frac{x}{2}) \\
     &\text{by Brigg's idenity} \\
     \implies \sum^n_{k=1} \text{sin}( (k+\frac{1}{2})x) - \text{sin}( (k-\frac{1}{2})x) & =\text{sin}( (2n+1) \frac{x}{2}) + \text{sin}(\frac{x}{2}) \\
     &\text{ Which telescopes to} \\
     - \text{sin}(\frac{-1}{2}x)+ \text{sin}( (2n+1) \frac{x}{2}) & = \text{sin}( (2n+1) \frac{x}{2}) + \text{sin}(\frac{x}{2}) \\
     \implies \text{sin}( (2n+1) \frac{x}{2}) + \text{sin}(\frac{x}{2}) & = \text{sin}( (2n+1) \frac{x}{2}) + \text{sin}(\frac{x}{2}) \\
\end{align*}
Thus, 
    \[\frac{\text{sin} \left( (2n+1)\frac{x}{2} \right) }{\text{sin} (\frac{x}{2}) }  =-1 + \sum^{n}_{k=1} 2 \text{cos}(x)\]
    \item Note, the critical point will be when $G_n'(x) = 0$. Thus, by the previous part,
        \begin{align*}
            \frac{\text{sin} \left( (2n+1)\frac{x}{2} \right) }{\text{sin} (\frac{x}{2}) } = 0\\
            \text{sin} \left( (2n+1)\frac{x}{2} \right) = 0 \quad \text{where sin$(x)\neq 0$}
        \end{align*}
        This will happen first when 
        \begin{align*}
        (2n+1) \frac{x}{2} & = \pi\\
        \implies x = \frac{2\pi}{(2n+1)}
        \end{align*}
\end{enumerate}



\exercise{4.6}\\
By the fundamental Theorem of Calculus, we know
\[
g_n(\theta_n) - g_n(0) = \int^{\theta_n}_0 \frac{\text{sin} \left( (2n+1)\frac{x}{2} \right)}{\text{sin} (\frac{x}{2})} dx
\]
Note, $g_n(0) = f(0) - \sum^\infty_{k=1} \frac{2}{k} \text{sin}(0) = \pi$, where $f(x)$ is the continous version of the function.
Thus, 
\[
g_n(\theta_n) = \int^{\theta_n}_0 \frac{\text{sin} \left( (2n+1)\frac{x}{2} \right)}{\text{sin} (\frac{x}{2})} dx - \pi
\]


\end{document}

