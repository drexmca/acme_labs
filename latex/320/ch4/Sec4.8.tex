\documentclass[letterpaper,12pt]{article}

\usepackage{threeparttable}
\usepackage{geometry}
\geometry{letterpaper,tmargin=1in,bmargin=1in,lmargin=1.25in,rmargin=1.25in}
\usepackage[format=hang,font=normalsize,labelfont=bf]{caption}
\usepackage{amsmath}
\usepackage{multirow}
\usepackage{array}
\usepackage{delarray}
\usepackage{amssymb}
\usepackage{amsthm}
\usepackage{lscape}
\usepackage{natbib}
\usepackage{setspace}
\usepackage{float,color}
\usepackage[pdftex]{graphicx}
\usepackage{mathrsfs}  
\usepackage{pdfsync}
\usepackage{verbatim}
\usepackage{placeins} \usepackage{geometry}
\usepackage{pdflscape}
\synctex=1
\usepackage{hyperref}
\hypersetup{colorlinks,linkcolor=red,urlcolor=blue,citecolor=red}
\usepackage{bm}
\usepackage{amssymb}


\theoremstyle{definition}
\newtheorem{theorem}{Theorem}
\newtheorem{acknowledgement}[theorem]{Acknowledgement}
\newtheorem{algorithm}[theorem]{Algorithm}
\newtheorem{axiom}[theorem]{Axiom}
\newtheorem{case}[theorem]{Case}
\newtheorem{claim}[theorem]{Claim}
\newtheorem{conclusion}[theorem]{Conclusion}
\newtheorem{condition}[theorem]{Condition}
\newtheorem{conjecture}[theorem]{Conjecture}
\newtheorem{corollary}[theorem]{Corollary}
\newtheorem{criterion}[theorem]{Criterion}
\newtheorem{definition}{Definition} % Number definitions on their own
\newtheorem{derivation}{Derivation} % Number derivations on their own
\newtheorem{example}[theorem]{Example}
\newtheorem*{exercise}{Exercise} % Number exercises on their own
\newtheorem{lemma}[theorem]{Lemma}
\newtheorem{notation}[theorem]{Notation}
\newtheorem{problem}[theorem]{Problem}
\newtheorem{proposition}{Proposition} % Number propositions on their own
\newtheorem{remark}[theorem]{Remark}
\newtheorem{solution}[theorem]{Solution}
\newtheorem{summary}[theorem]{Summary}
\bibliographystyle{aer}
\newcommand\ve{\varepsilon}
\renewcommand\theenumi{\roman{enumi}}

\title{Sec 4.1 \\Math 320}
\author{Rex McArthur}

\begin{document}

\maketitle
\exercise{4.37}\\

\exercise{4.38}\\
Note, $f(x) = -2 \phi (4x) + 4 \phi(4x-1) + 2 \phi(4x-2) - 3 \phi(4x-3) \in V_2$\\ 
By the lemmas in the book, we can say that 
\begin{align*}
    f(x) & = -(\phi(2x) + \psi(2x)) + 2(\phi(2x) - \psi(2x)) \\
    &+ (\phi(2x-1) + \psi(2x-1) ) - \frac{3}{2}(\phi(2x-1) - \psi(2x-1) ) \\
    & = \psi(2x) - 3 \psi(2x) - \frac{1}{2} \phi(2x-1) + \frac{5}{2} \psi(2x-1) \\
    & = \frac{1}{2} (\phi(x) + \psi) - 3 \psi(2x) - \frac{1}{4}(\phi(x) - \psi(x) ) + \frac{5}{2}-\psi(2x-1) \\
    & = \frac{1}{4}\phi(x) + \frac{3}{4}\psi(x) + (-3\psi(2x)+ \frac{5}{2} \psi(2x-1) )
\end{align*}

Where we know that 
\begin{align*}
    &\frac{1}{4} \phi(x) \in V_0\\
    &\frac{3}{4}\psi(x) \in W_0\\
    &-3\psi(2x) + \frac{5}{2} \psi (2x-1) ) \in W_1
\end{align*}
    

\exercise{4.39}\\

\exercise{4.40}\\





\end{document}

