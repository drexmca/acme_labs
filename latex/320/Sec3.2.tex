\documentclass[letterpaper,12pt]{article}

\usepackage{threeparttable}
\usepackage{geometry}
\geometry{letterpaper,tmargin=1in,bmargin=1in,lmargin=1.25in,rmargin=1.25in}
\usepackage[format=hang,font=normalsize,labelfont=bf]{caption}
\usepackage{amsmath}
\usepackage{multirow}
\usepackage{array}
\usepackage{delarray}
\usepackage{amssymb}
\usepackage{amsthm}
\usepackage{lscape}
\usepackage{natbib}
\usepackage{setspace}
\usepackage{float,color}
\usepackage[pdftex]{graphicx}
\usepackage{mathrsfs}  
\usepackage{pdfsync}
\usepackage{verbatim}
\usepackage{placeins} \usepackage{geometry}
\usepackage{pdflscape}
\synctex=1
\usepackage{hyperref}
\hypersetup{colorlinks,linkcolor=red,urlcolor=blue,citecolor=red}
\usepackage{bm}
\usepackage{amssymb}


\theoremstyle{definition}
\newtheorem{theorem}{Theorem}
\newtheorem{acknowledgement}[theorem]{Acknowledgement}
\newtheorem{algorithm}[theorem]{Algorithm}
\newtheorem{axiom}[theorem]{Axiom}
\newtheorem{case}[theorem]{Case}
\newtheorem{claim}[theorem]{Claim}
\newtheorem{conclusion}[theorem]{Conclusion}
\newtheorem{condition}[theorem]{Condition}
\newtheorem{conjecture}[theorem]{Conjecture}
\newtheorem{corollary}[theorem]{Corollary}
\newtheorem{criterion}[theorem]{Criterion}
\newtheorem{definition}{Definition} % Number definitions on their own
\newtheorem{derivation}{Derivation} % Number derivations on their own
\newtheorem{example}[theorem]{Example}
\newtheorem*{exercise}{Exercise} % Number exercises on their own
\newtheorem{lemma}[theorem]{Lemma}
\newtheorem{notation}[theorem]{Notation}
\newtheorem{problem}[theorem]{Problem}
\newtheorem{proposition}{Proposition} % Number propositions on their own
\newtheorem{remark}[theorem]{Remark}
\newtheorem{solution}[theorem]{Solution}
\newtheorem{summary}[theorem]{Summary}
\bibliographystyle{aer}
\newcommand\ve{\varepsilon}
\renewcommand\theenumi{\roman{enumi}}

\title{Sec 3.2 \\Math 320}
\author{Rex McArthur}

\begin{document}
\maketitle
\exercise{3.7}\\

\[ ||\text{proj}x(v) ||^2 = \sum^{m}_{i=1} \left| \frac{\langle x_i,v \rangle}{||x_i||}\right|^2 \leq ||v||^2\]
\[\implies \sum^{m}_{i=1}  \frac{|\langle x_i,v \rangle|^2}{||x_i||^2} \leq ||v||^2\]
\[\implies \sum^{m}_{i=1}  |\langle x_i,v \rangle|^2 \leq ||v||^2||x_i||^2\]
\[\implies \sum^{m}_{i=1}  |\langle x_i,v \rangle| \leq ||v||||x_i||\]
\[\implies   |\langle x_i,v \rangle| \leq ||v||||x_i||\]

\exercise{3.8}\\
(i)
You can show that is normal, but showing the length of each is one, and the inner product of each combo = 0. Note,
\[\langle  cos(x), cos(x)\rangle   = \frac{1}{\pi}\int ^{\pi}_{-pi} cos^2(x)dx = \frac{1}{\pi}[x+sin(x)cos(x)]^\pi_0 = 1\]
\[\langle  sin(x), sin(x)\rangle   = \frac{1}{\pi}\int ^{\pi}_{-pi} sin^2(x)dx = \frac{1}{\pi}[x-sin(x)cos(x)]^\pi_0 = 1\]
Thus, they are each normal. \\
The other two follow immediatly from above, because the only difference is the period. To show orthogonality, 
\begin{align*}
\langle  cos(x), sin(x)\rangle   &= \frac{1}{\pi}\int ^\pi_{-\pi} cos(x)sin(x)dx = (\frac{1}{\pi})(\frac{-1}{2}) cos^2(x)_{-\pi}^\pi = 0 \\
\langle  cos(x), cos(2x)\rangle   &= \frac{1}{\pi}\int_{-\pi}^{\pi}cos(x)cos(2x)dx=\frac{1}{\pi}\int_{-\pi}^{\pi}cos(x)[cos^2(x)-sin^2(x)]dx \\
&=\frac{1}{\pi}\int_{-\pi}^{\pi}cos^3(x)dx-\int_{-\pi}^{\pi}sin^2(x)cos(x)dx = \frac{1}{\pi}(0)=0 \\
\langle  cos(x), sin(2x) \rangle   &= \frac{1}{\pi}\int_{-\pi}^{\pi}cos(x)sin(2x)=\frac{1}{\pi}[\frac{-2}{3}cos^3(x)]_{-\pi}^\pi=0 \\
\langle  sin(x), cos(2x) \rangle   &= \frac{1}{\pi}\int_{-\pi}^{\pi}cos^2(x)sin(x)dx-\int_{-\pi}^{\pi}sin^3(x)dx= -\frac{1}{3\pi}cos^3(x)_{-\pi}^\pi = 0 \\
\langle  sin(x),sin(2x)\rangle  &= \frac{1}{\pi}\int_{-\pi}^{\pi}sin(x)sin(2x)dx=\frac{2}{3\pi}sin^3(x)_{-\pi}^\pi = 0 \\
\end{align*}
Again, the result for the last two follow from these, because the only difference is a change in the period.\\
(ii)\\
\[\|t\| = \langle  t,t \rangle   = \frac{1}{\pi}\int_{-\pi}^{\pi}t^2dt = \frac{1}{3\pi}[t^3]^\pi_{-\pi}=\frac{2\pi^2}{3}\]
(iii)\\
\begin{align*}
    \text{proj}_X(cos(3t)) &= \langle  cos(t),cos(3t)\rangle   cos(t)+\langle  sin(t),cos(3t)\rangle   sin(t)+\\ 
&~~~~~\langle  (cos(2t),cos(3t)\rangle   cos(2t) +\langle  sin(2t),cos(3t)\rangle   sin(2t)\\
&=\frac{1}{\pi}\bigg(\int^\pi_{-\pi}cos(t)cos(3t)dt ~cos(t)+\int^\pi_{-\pi}sin(t)(cos(3t)dt ~sin(t)+\\ 
&~~~~~~~~\int^\pi_{-\pi}cos(2t)cos(3t)dt~cos(2t)+ \int^\pi_{-\pi}sin(2t)cos(3t)dt~sin(2t)\bigg)\\
&=\frac{1}{\pi}\big(0 \cdot cos(t)+\frac{-3\pi}{2}sin(t)+0+4\pi sin(2t)\big)=\frac{-3}{2}sin(t)+ 4sin(2t)
\end{align*}
This is a linear combination of the starting equations.\\ \\
(iv)\\
\begin{align*}
    \text{proj}_{x}(t) &= \langle cos(t),t\rangle cos(t)+\langle  sin(t),t\rangle sin(t) + \langle  (cos(2t),t\rangle   cos(2t) +\langle  sin(2t),t\rangle   sin(2t)\\
&=\frac{1}{\pi}\bigg(\int^\pi_{-\pi}cos(t)t~dt ~cos(t)+\int^\pi_{-\pi}sin(t)t~dt ~sin(t)+\\ 
&~~~~~~~~\int^\pi_{-\pi}cos(2t)t~dt~cos(2t)+ \int^\pi_{-\pi}sin(2t)t~dt~sin(2t)\bigg) \\
&=\frac{1}{\pi}(2\pi sin(t)-\pi sin(2t))=2sin(t)-sin(2t)
\end{align*}
This is a linear combination of the starting equations.

\exercise{3.9}\\

Consider the counterexample of two unit vectors $x,y \in V$ that are orthogonal to $X$. In this case, $\text{proj}X(y)= 0 \neq 1 \quad \text{proj}X(x)= 0 \neq 1 $ and we have the desired result, because orthonormal transformations perserves vector length.

\exercise{3.10}\\



If we rotate two vectors $x,y$ by $\theta$, we preserve the angle between two vectors since we rotate them both by $\theta$. 
Also, length is unaffected by rotation. and not performing other operations on them. Therefore, this is an orthonormal transformation.
\exercise{3.11}\\
(i)\\
By properties of orthogonality,
\[   \|Qx\|_{2} =  \langle Qx,Qx \rangle = (Qx)^HQx = x^HQ^HQx = x^HIx = x^Hx = \langle x,x\rangle = \|x\|_{2}  \]
(ii)\\
By (v) 
\[(Q_{1}Q_{2})(Q_{1}Q_{2})^H = Q_{1}Q_{2}Q_{2}^HQ_{1}^H= Q_{1}Q_{1}^{-1} = I \]
which implies orthogonality.
(iii)\\
Because Q is orthonormal, when we take the Hermetian of the inverse this results in the original Q allowing us to perform the following proof:
\[\langle Q^{-1},Q^{-1}\rangle = Q^{-1}(Q^{-1})^H = Q^{-1}Q = I\]
(iv)\\
$\Rightarrow$ By (iii), if Q is orthonormal, the resulting statement is true. 
\[Q^HQ = Q^{-1}Q= QQ^{-1} = I\]
$\Leftarrow$ If $Q^HQ=QQ^H=Q^{-1}Q=QQ^{-1}=I$ implying orthonormality yielding that $\|Q\| = 1$ and is orthonormal. \\ \\
(v)\\
Because we know that if Q is orthonormal then either the rows or the columns must be orthonormal. Suppose that the rows are orthonormal. Because we know that $Q^{T}$ is also orthonormal then the columns of Q are orthonormal. \\ \\
(vi)\\
We know that $Q^{-1} = Q \implies Q^{-1}Q = I \implies \text{det}(Q^{-1}Q) = \text{det}(I) = \text{det}(Q^{-1}) \text{det}(Q) = \text{det}(I) \implies \text{det}(Q^H) \text{det}(Q) = \text{det}(I)\implies \text{det}(Q) \text{det}(Q) = 1 \implies \text{det}(Q) = \pm 1 \implies |\text{det}(Q) | = 1$
\\\\The converse is not true. Take the matrix $A$ as an example where
\[
\begin{bmatrix}
    \frac{1}{2}&0 \\
    0&2 \\
\end{bmatrix}
\]
Whose determinant is 1 but whose columns are not orthonormal.

\end{document}

